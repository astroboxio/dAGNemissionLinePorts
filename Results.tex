\section{Results}

\subsection{Comparative Analyses}
 \subsubsection{Population vs. Control Results: $\text{[O II]}/\text{[O III]}$}

In order to better understand the star-formation properties of dual AGN we select three distinct groups to analyse. Firstly, we have our control sample of Type II AGN with $\log{(\text{[O III]}/\text{H}\beta)}>{6.9}$; which we feel can act as archetypal narrow-line AGN sample from which we gain gain context for the Type II candidate dual AGN sample. Coupled to the two main samples, we also add a third: tidal\footnote{\tiny{Objects classified as tidally interacting taken from \cite{Liu_2011}.}} (interacting), binary\footnote{\tiny{kpc-scale binary AGNs of comparable luminosities, with a relative orbital velocity $\gtrsim{150}$ km s$^{-1}$ taken from \cite{Shen_Liu_Greene_Strauss_2011}.}} and visually confirmed dual-cored AGN candidates taken from \cite{Ge_Hu_Wang_Bai_Zhang_2012}, and that coincide and can be cross-referenced with objects within our Liu+2010 parent dual AGN sample. With this in mind, we firstly look towards the amount of the $\text{[O II]}$ $\lambda\lambda$$3727.09$, $3729.88$\AA\AA{} and $\text{H}\alpha$ integrated flux (in units of $1\text{E}^{-17}$ erg s$^{-1}$ cm$^{-2}$) relative to the integrated flux of the $\text{[O III]}$ $\lambda$$5008.24$\AA. These ratios are important in both our understanding of the star-formation properties of the host galaxies of dual AGN, and the contamination of these particular-emission lines from the dual AGN contribution.

Our analysis of the $\text{[O II]/[O III]}$ ratio for single Type II AGN sample yielded the following statistical results: a 2nd quartile i.e., median, $\tilde{x}$ of 0.2318; standard deviation $\sigma_{X}$ of 0.1402 with a cleaned mean, $\mu$, (i.e., excluded outliers within one standard deviation of the median) of 0.1727 and an uncleaned mean of 0.2681. We also report an average absolute deviation around the median of 0.1030, and minimum and maximum values of 0.0836 and 0.7053, respectively. For a desired confidence level of 95\% for the mean quoted, the corresponding confidence interval is ${0.2403}<{\mu}<{0.2956}$. However, if we exclude all \textit{Tukey outliers} (i.e., those objects which are greater than 1.5 times the interquartile range (IQR) above the third quartile or below the first quartile) for this sample we can quote a 95\% confidence level of ${0.2311}<{\mu}<{0.2809}$.

We applied the same statistical analyses for our Liu+2010 candidate dual AGN sample and to our tidal, binary and dual-cored AGN candidates. Firstly, our analysis of the Type II AGN with double-peaked emission-lines sample yielded the following statistical results: a $\tilde{x}$ of 0.3354; $\sigma_{X}$ of 0.2541 with a $\mu_{c}$ of 0.2771 and a $\mu$ of 0.3988. We also report an average absolute deviation around the median of 0.1450, and minimum and maximum values of 0.1125 and 2.1492, respectively. For a desired confidence level of 95\% for the mean quoted, the corresponding confidence interval is ${0.3584}<{\mu}<{0.4407}$. Further to our statistical analyses for our Type II AGN with double-peaked emission-lines sample, if we were to isolate those objects within our parent sample designated as tidally interacting, binary or dual-cored AGN we quote a $\tilde{x}$ of 0.4594; $\sigma_{X}$ of 0.3041 with a $\mu_{c}$ of 0.3504 and a $\mu$ of 0.5241. We also report an average absolute deviation around the median of 0.2110, with minimum and maximum values of 0.2319 and 1.3832, respectively. Again, for a desired confidence level of 95\% for the mean quoted, the corresponding confidence interval is ${0.3775}<{\mu}<{0.6707}$.

Comaparing all three samples it is evident that within the candidate dual AGN sample there are objects which have elevated $\text{[O II]}$ emission relative to our pure Type II AGN control sample, where our dual AGN which are classified in the literature as either tidally interacting, binary or dual-cored are again showing signs of elevated $\text{[O II]}$ emission relative to both the Type II AGN and the candidate Type II dual AGN samples.

\subsection{Star Formation Rates, Stellar Ages \& $L_{\text{[O II]}}$}

In this section we address the main topic of this work, i.e. the comparisons of the general star formation and stellar age properties of candidate Type II dual AGN host galaxies with Type II AGN galaxies of similar mass. We begin by calculating the average luminosity of the \text{[O II]} flux, and using this value to obtain an estimate of the specific SFR via \label{eq:Kewley} and measured host stellar masses, for both our control and double-peaked sample in bins of $\text{D}_n(4000)$. This process serves as a test of our estimated SFR from the $\text{[O II]}$ method outlined in §3.2.1, as well as a benchmark test between our known control population of pure Type II AGN (i.e., $\log{(\text{[O III]}/\text{H}\beta)}>{0.69}$) and our double-peaked candidate dual Type II AGN sample. By design, this diagnostic serves as a proxy test between the instantaneous SFR and the mean stellar age and subsequent star formation history, by forming a semi-empirical relationship between the $4000Å$ break and the specific star-formation rate (as shown in \ref{Brinchmann}).   

For our candidate Type II dual AGN sample our $\text{[O II]}$ luminosities range from $\sim{8\times{10}^{39}}$ to $4.2\times{10}^{42}$ erg s$^{-1}$, with an average $\text{[O II]}$ luminosity, $\left\langle{L_{\text{[O II]}}}\right\rangle$, of $\sim{3.5\times{10^{41}}}$ erg s$^{-1}$. Applying the maximum plausible AGN contamination extent, as calculated from the statistical analysis of the $\text{[O II]}/\text{[O III]}$ ratio applied to our control sample (as outlined §4.1.1) as well as metallicity and extinction corrections yields an average SFR of $\sim{1.45}M_{\odot}$. We also find that the average $\text{[O III]}$ log  luminosity, $\left\langle\log{L_{\text{[O III]}}}\right\rangle$, of $\sim{41.78}$. As the $\text{[O III]}$ luminosity is often assumed to be a very good indication of AGN power, we note that there is a tendency of the most powerful of the dual AGN candidates to play host to younger stellar populations, where the average $\text{[O III]}$ luminosity for those objects with a measured $\text{D}_n(4000)$ index $<{1.2}$ is of the order of $\sim{10^{42.6}}$, compared with $\sim{10^{41}}$ for object with average older stellar populations (i.e., $\text{D}_n(4000)>{1.5}$).         
 
\subsubsection{Kolmogorov-Smirnov (K-S) Test between Type II AGN and the candidate Type II dual AGN.}

Given the uncertainties and the spread in the value for the major emission-lines $\text{[O II]}$ and $\text{[O III]}$ luminosity, we conducted a non-parametric Kolmogorov-Smirnov (K-S) test [\cite{Peacock_1983}; \cite{Justel_1997}]. This test can be used to examine any potential variability between our dual AGN sample and our control AGN sample as well as to evaluate whether the two Type II narrow-line samples belong to two distinct and different populations (even with a potential large amount of the double-peaked emission line profiles within our candidate dual AGN sample being attributed to outflows, rotating gaseous discs on kpc scales or asymmetrical jets emerging from the central engine). As such, if we assume that such double-peaked emission lines which produce such phenomena can potentially be seen in normal narrow-line Type II AGN we should not see any distinct differences between the two samples, i.e., we should not see an elevated level of star-formation. In essence, the K-S test can help us determine if our two data samples come from the same underlying distribution. We perform the K-S test by using the ad hoc python module, scipy.stats.ks_2samp. We adopt a null hypothesis, $\text{H}_{0}$, whereby the $\text{[O II]}/\text{[O III]}$ two samples are considered to be generically of the same population.  

In \ref{KolmogorovSmirnov} we can see an empirical cumulative distribution function  performed for the $\text{[O II]}/\text{[O III]}$ line ratio for both of our samples as well as a binned distribution histogram for both samples. For our chosen samples, i.e., $n_{1}=n_{\text{AGN}}=100$ and $n_{2}=n_{\text{dAGN}}=148$, the K-S statistic has a value of $0.38$, far greater than the above boundary; we can reject the $\text{H}_{0}$ with a \textit{p-value} of $5.49\times10^{-8}$. This is a clear indication that the two samples came from two different populations. Underlining the diverse nature of the $\text{[O II]}/\text{[O III]}$ line ratio for the two samples, we measure a skewness (i.e., asymmetry of the probability distribution) value of 1.18 for our Type II AGN control sample and a value of 3.54 for our candidate double-peaked Type II dual AGN sample. Essentially, the histogram plot confirms this asymmetry test where the candidate double-peaked Type II dual AGN sample has a higher number of outliers and objects exceeding the $\mu+{2\sigma}$.