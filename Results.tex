\section{Results \& Discussions}
\subsection{Sample Population}
\subsection{Ge+2012 Type II Sub-Sample Global SFR}

From the initial sample of 3841 objects which are classified as Ge+2012 Type 2 objects (i.e., binary Seyfert Type II galaxies) we obtained spectroscopic measurements from the newly released DR10 as calculated by the Portsmouth Group [see: \cite{Thomas_2013}] for all major emission-lines attributable to star-forming and active galaxies. Using the methods outlined in §3.1.2, we calculated star-formation rates for all those objects within the initial sample that had been re-analysed by the Portsmouth Group, a total of 3348 cross-referenceable objects (with one object removed, namely SDSS J124053.63+010030.0 due to erroneous spectroscopic data). We also supplemented both the measurements conducted by \cite{Ge_Hu_Wang_Bai_Zhang_2012} and the emission line properties \cite{Thomas_2013} with calculations of the $\log{(\text{[O II] }{\lambda}3727Å\text{/[O III]})}$ ratio as well as approximations for the $L_{\text{[O III]}}$ and $L_{\text{[O II]}}$ luminosities and a global bolometric luminosity estimation for low-redshift Type II AGN using the $\text{[O I]}\lambda6363Å/\text{[O III]}$ ratio taken from \cite{Netzer_2009}.         

As alluded to in §3.1.2 an initial proxy of the star-formation rate (SFR) uncorrected for an AGN contribution, metallicity, dust obscuration and reddening was calculated for the entirety of the Ge+2012 DR7 Type II AGN sub-sample using the method outlined in \cite{Kennicutt_1998} using spectroscopic measurements from DR10. Using the basic \cite{Kennicutt_1998} method we obtained a median SFR value of $16.705\text{M}_{\odot}$ yr$^{-1}$. Further utilising the method of \cite{Kewley_2004}, correcting for an AGN contribution advised from \cite{2006ApJ...642..702K} as well as corrections arising from metallicity, dust obscuration and reddening, a median AGN correction SFR of $6.221\text{M}_{\odot}$ yr$^{-1}$ was calculated. We also find from the overall parent sample of DR7 Type II objects that approximately 30\% of objects correspond to SFRs greater than the accepted value for star-forming galaxies of $10\text{M}_{\odot}$ yr$^{-1}$, comparable to that of the star-forming galaxy M82 (the ‘Cigar’ galaxy) [\cite{2009ApJ...706.1364F}]. Of the 30\% that could be classified as star-forming galaxies 35\% have corresponding SFR values greater than $20\text{M}_{\odot}$ yr$^{-1}$, a figure usually associated with starburst galaxies. As a benchmark, of the total sample we also find enhanced SFRs for 2072 objects (61.9\%) greater than 5 M$_{\odot}$ yr$^{-1}$ corresponding to two times the SFR of our own Milky Way galaxy (i.e., $1.9\pm{0.4}\text{M}_{\odot}$ yr$^{-1}$ [\cite{2011AJ....142..197C}]). Further disassociating our sample into the three main BPT classifications (neglecting LINERs) i.e., AGN, composite galaxies and star-forming galaxies we have median SFR values of $5.64\text{M}_{\odot}$ yr$^{-1}$, $6.66\text{M}_{\odot}$ yr$^{-1}$ and $6.75\text{M}_{\odot}$ yr$^{-1}$, respectively. 

The median $\log{(\text{[O II] }{\lambda}3727Å\text{/[O III]})}$ value for those objects classified as Type II binary AGN candidates from the Ge+2012 sample 0.4849, taking the OII doublet spectroscopic measurements from the Portsmouth Group. The median $\log{(\text{[O II] }{\lambda}3727Å\text{/[O III]})}$ values for the reclassified DR10 sample for AGN, composite galaxies and star-forming galaxies are -0.1341, 0.5431 and 0.6929, respectively.      