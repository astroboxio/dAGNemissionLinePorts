\section{Results \& Discussions}

\subsection{Comparative Analyses}

\subsection{Star Formation Rates, Stellar Ages \& $L_{\text{[O II]}}$}

In this section we address the main topic of this work, i.e. the comparisons of the general star formation and stellar age properties of candidate Type II dual AGN host galaxies with Type II AGN galaxies of similar stellar mass. We begin by calculating the average luminosity of the \text{[O II]} flux, and using this value to obtain an estimate of the specific SFR via \label{eq:Kewley} and measured host stellar masses, for both our control and double-peaked sample in bins of $\text{D}_n(4000)$. This process serves as a test of our estimated SFR from the $\text{[O II]}$ method outlined in §3.2.1, as well as a benchmark test between our known control population of pure Type II AGN (i.e., $\log{(\text{[O III]}/\text{H}\beta)}>{0.69}$) and our double-peaked candidate dual Type II AGN sample. By design, this diagnostic serves as a proxy test between the instantaneous SFR and the mean stellar age and subsequent star formation history, by forming a semi-empirical relationship between the $4000Å$ break and the specific star-formation rate, $\text{SFR}/M_{*}$, (as shown in \ref{Brinchmann}). We also plot the combined luminosity of the double-peaked $\text{[O III]}$ emission-line against the $\log{\text{[O II]}/\text{[O III]}}$, essentially treating the dual AGN as a singular AGN object. The rationale for treating the $\text{[O III]}$ emission is related to the measurement of the stellar mass, $M_{*}$, the $\text{[O II]}$ emission and the $4000Å$ break. Measurements of these properties can only be estimated for the entire galaxy containing the dual SMBH, rather than being resolved for each individual AGN. Hence, any comparisons will be considered for the aggregate of AGN spectral measurements.     

From \ref{Brinchmann} we find that there are both similarities within our candidate Type II dual AGN sample relative to our control sample and many differences. It is evident that there is a general elevation of star-formation rates for our candidate dual AGN double-peaked Type II AGN sample, as well as a precedent for younger stellar populations. However, the SFR has no real relation to the level of OII relative to the $\text{[O III]}$ flux. This “perceived” elevation in the OII emission-line may well be due to triggering timings of AGN within merging galaxies. The flux of the $\text{[O III]}$ emission-line scales well with AGN activity; where the peak of AGN accretion should correspond to the peak $\text{[O III]}$ luminosity. The bolometric luminosity, $L_{\text{bol}}$, has also been shown to be in phase with each pericentre passage; triggering at the point of closest SMBH approach and hitting peak luminosity at SMBH separations $<{10}$ kpc. Hence, if the AGN merger is not at this peak triggering phase then the luminosity of the $\text{[O III]}$ emission may well be depressed. If, as is highly plausible, the triggering of peak accretion onto the central SMBH pair is not in phase with the peak of star-formation then observing an elevated $\text{[O II]}$ flux relative to the $\text{[O III]}$ emission may be common, and not an indication of the actual current SFR. What is noticeable, however, is that the candidate dual Type II AGN represent galaxies with a statistically younger stellar population, with a mean $\text{D}_{n}(4000)Å$ break index of $1.394$ compared to a mean $\text{D}_{n}(4000)Å$ break index of $1.467$ for our Type II single AGN control sample.     

For our candidate Type II dual AGN sample our $\text{[O II]}$ luminosities range from $\sim{8\times{10}^{39}}$ to $4.2\times{10}^{42}$ erg s$^{-1}$, with an average $\text{[O II]}$ luminosity, $\left\langle{L_{\text{[O II]}}}\right\rangle$, of $\sim{3.5\times{10^{41}}}$ erg s$^{-1}$. Applying the maximum plausible AGN contamination extent, as calculated from the statistical analysis of the $\text{[O II]}/\text{[O III]}$ ratio applied to our control sample (as outlined §4.1.1) as well as metallicity and extinction corrections yields an average SFR of $\sim{1.45}M_{\odot}$. We also find that the average $\text{[O III]}$ log  luminosity, $\left\langle\log{L_{\text{[O III]}}}\right\rangle$, of $\sim{41.78}$. As the $\text{[O III]}$ luminosity is often assumed to be a very good indication of AGN power, we note that there is a tendency of the most powerful of the dual AGN candidates to play host to younger stellar populations, where the average $\text{[O III]}$ luminosity for those objects with a measured $\text{D}_n(4000)$ index $<{1.2}$ is of the order of $\sim{10^{42.6}}$, compared with $\sim{10^{41}}$ for object with average older stellar populations (i.e., $\text{D}_n(4000)>{1.5}$). This is consistent with the findings of \cite{Kauffmann_2003} that the most luminous AGN are playing host to younger stellar ages. It is evident that a number of dual AGN candidates are not only in a luminous phase of their accretion event, but also undergoing enhanced star-formation; and it may also be suggestive that only the most gas rich and massive galaxy mergers can supply concurrent dual AGN activity with ongoing star-formation. We also note, that although there are major differences between our control pure AGN sample (i.e., AGN with $$).  

Evident in our $\log{L_{\text{[O III]}}}$ vs. $\log{\text{[O II]}/\text{[O III]}}$ plot we see that the most luminous AGN at $\lambda5007Å$ (i.e., $\log{\text{[O II]}/\text{[O III]}}>{10^{42}}$) also have much lower $\text{[O II]}/\text{[O III]}$ ratios relative to the whole sample. Interestingly, and perhaps contradictory to this, is that the aggregate $\log{L_{\text{[O III]}}}$ for the most powerful AGN is concurrent with younger stellar ages yet spans quite large $M_{*}$ and $\log{\text{SFR}_{\text{O II}}/M_{*}}$ ranges.
 
\subsubsection{Kolmogorov-Smirnov (K-S) Test between Type II AGN and the candidate Type II dual AGN.}

Given the uncertainties and the spread in the value for the major emission-lines $\text{[O II]}$ and $\text{[O III]}$ luminosity, we conducted a non-parametric Kolmogorov-Smirnov (K-S) test [\cite{Peacock_1983}; \cite{Justel_1997}]. This test can be used to examine any potential variability between our dual AGN sample and our control AGN sample as well as to evaluate whether the two Type II narrow-line samples belong to two distinct and different populations (even with a potential large amount of the double-peaked emission line profiles within our candidate dual AGN sample being attributed to outflows, rotating gaseous discs on kpc scales or asymmetrical jets emerging from the central engine). As such, if we assume that such double-peaked emission lines which produce such phenomena can potentially be seen in normal narrow-line Type II AGN we should not see any distinct differences between the two samples, i.e., we should not see an elevated level of star-formation. In essence, the K-S test can help us determine if our two data samples come from the same underlying distribution. We perform the K-S test by using the ad hoc python module, scipy.stats.ks_2samp. We adopt a null hypothesis, $\text{H}_{0}$, whereby the $\text{[O II]}/\text{[O III]}$ two samples are considered to be generically of the same population.  

In \ref{KolmogorovSmirnov} we can see an empirical cumulative distribution function  performed for the $\text{[O II]}/\text{[O III]}$ line ratio for both of our samples as well as a binned distribution histogram for both samples. For our chosen samples, i.e., $n_{1}=n_{\text{AGN}}=100$ and $n_{2}=n_{\text{dAGN}}=148$, the K-S statistic has a value of $0.38$, far greater than the above boundary; we can reject the $\text{H}_{0}$ with a \textit{p-value} of $5.49\times10^{-8}$. This is a clear indication that the two samples came from two different populations. Underlining the diverse nature of the $\text{[O II]}/\text{[O III]}$ line ratio for the two samples, we measure a skewness (i.e., asymmetry of the probability distribution) value of 1.18 for our Type II AGN control sample and a value of 3.54 for our candidate double-peaked Type II dual AGN sample. Essentially, the histogram plot confirms this asymmetry test where the candidate double-peaked Type II dual AGN sample has a higher number of outliers and objects exceeding the $\mu+{2\sigma}$.
  
  
  
  
  
  
  
  