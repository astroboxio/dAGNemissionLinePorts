\section{Results}

\subsection{Comparative Analyses}

\subsubsection{Star Formation Rates, Stellar Ages \& $L_{\text{[O II]}}$}

In this section we address the main topic of this work, i.e. the comparisons of the general star formation and stellar age properties of candidate Type II dual AGN host galaxies with Type II AGN galaxies of similar stellar mass. We begin by calculating the average luminosity of the \text{[O II]} flux, and using this value to obtain an estimate of the specific SFR via \label{eq:Kewley} and measured host stellar masses, for both our control and double-peaked sample in bins of $\text{D}_n(4000)$. This process serves as a test of our estimated SFR from the $\text{[O II]}$ method outlined in §3.2.1, as well as a benchmark test between our known control population of pure Type II AGN (i.e., $\log{(\text{[O III]}/\text{H}\beta)}>{0.69}$) and our double-peaked candidate dual Type II AGN sample. By design, this diagnostic serves as a proxy test between the instantaneous SFR and the mean stellar age and subsequent star formation history, by forming a semi-empirical relationship between the $4000$Å break and the specific star-formation rate, $\text{SFR}/M_{*}$, (as shown in \ref{Brinchmann}). We also plot the combined luminosity of the double-peaked $\text{[O III]}$ emission-line against the $\log{\text{[O II]}/\text{[O III]}}$ ratio, treating our dual AGN objects as a singular AGN object. The rationale for treating the $\text{[O III]}$ emission in this way is related to the measurement of the stellar mass, $M_{*}$, the $\text{[O II]}$ emission and the $4000$Å break. Measurements of these properties can only be estimated for the entire galaxy containing the dual SMBH, rather than being resolved for each individual AGN. Hence, any comparisons will be considered for the aggregate of AGN spectral measurements.

Utilising \ref{eq:KimCorrection} and \ref{eq:Kewley} we measured the AGN corrected SFR for our double-peaked candidate Type II dual AGN sample. We find that the $\text{SFR}$ of host dual AGN host galaxies are within a confined range of $0.01-20$ $M_{\odot}$ yr$^{-1}$ and where $\sim${33\%} have SFRs greater than $1$ $M_{\odot}$ yr$^{-1}$. Although this is slightly higher than our pure AGN control sample (where 23\% of objects play host to SFR greater than $1$ $M_{\odot}$ yr$^{-1}$), we note that previous work [e.g., \cite{Kauffmann_2003}; \cite{Silverman_2009}] find that for low-redshift objects sit nicely in a SFR range of $0.01-10$ $M_{\odot}$ yr$^{-1}$ and higher-redshift objects within a SFR range almost one order of magnitude higher of $0.1-100$ $M_{\odot}$ yr$^{-1}$. These results can be interpreted in different ways. We first note that our dual AGN sample may contain a large number of AGN in which the double-peaked emission-line profile can be explain without the need for binary or dual SMBH accretion [e.g., \cite{2011ApJ...727...71F}], such as rotating gaseous discs on kpc scales, asymmetrical jets or biconical outflows. Hence, although the $\text{SFR}$ range of our candidate dual AGN sample and our classic AGN control sample were similar it may be due to contamination of single AGN within our candidate dual AGN sample that are experiencing these complex disk kinematics and are more akin to the AGN which are considered to be due to a single SMBH. Secondly, and we proceed with caution with this interpretation, but could the major type II AGN samples [\cite{Kauffmann_2003}; \cite{2006AJZakamska1496Z}; \cite{Gu_2006}; \cite{Silverman_2009}] contain a large number of either resolvable dual AGN or AGN that have undergone a recent merger event. Such AGN could be experiencing heightened star-formation but whose double-peaked emission profile is absent due to either asymmetric triggering or offset AGN [e.g., \cite{2013ApJ...777...64C}; \cite{2015ApJ...806..219C}] or the double-peaked emission-line profile being unresolvable by SDSS standards due to the SMBHs being at close separations (e.g., approximately $<{1.5}$ kpc).   

In order to compliment measurements of the $\text{SFR}_{\text{[O II]}}$ for our candidate dual AGN sample we provide measurements of the stellar age of our sample utilising the narrower definition of the $\text{D}_{n}(4000)$ spectral index [\cite{1999ApJ...527...54B}] and the relation given by \cite{Kauffmann_2003} which infers the stellar age from the $\text{D}_{n}(4000)$ spectral index by using an instantaneous burst model of solar-metallicity. The break at $4000$Å is the strongest discontinuity in the optical spectrum of a galaxy [\cite{Bruzual_A__1983}; \cite{1999ApJ...527...54B}; \cite{Kauffmann_2003}]. The $\text{D}_{n}(4000)$ break index, i.e., the ratio of the average flux density, $\text{F}_{\nu}$, in the red continuum ($4001$--$4103$Å) to that in the blue continuum ($3849$--$3952$Å) [see \cite{1999ApJ...527...54B}; \cite{Kauffmann_2003}] utilises narrower continuity bands than the $\text{D}(4000)$ index [\cite{Bruzual_A__1983}], where the recent narrow definition is considerably less sensitive to reddening effects. For mean younger stellar ages (metal-poor) galaxies the expected $D_{n}(4000)$ spectral index is $<{1.5-1.6}$ where galaxies at the upper range of this index represent older galaxies with little to no recent or major star-formation present in their spectra. Conversely, those galaxies at the lower end of the range, i.e., ${0.2}<{\text{D}_{n}(4000)}<{1.5}$, are therefore consistent with galaxies with a younger mean stellar age, where a fraction of the total stellar mass has formed $\sim{0.1-2}$ $\text{Gyr}$ ago. For the entirety of this paper we assume Gaussian errors measured for the measured $\text{D}_{n}(4000)$. In \ref{Brinchmann} we show the values of the $\text{D}_{n}(4000)$ spectral index in relation to the specific star-formation rate (sSFR) i.e., $\log{(\text{SFR}_{\text{[O II]}}/M_{*})}$. We also demarcate our data according to young ($\text{D}_{n}(4000)<{1.2}$), intermediate (${1.2}<{\text{D}_{n}(4000)}<{1.5}$) and older ($\text{D}_{n}(4000)>{1.5}$) stellar populations. Using the derived relationships between stellar ages and the $\text{D}_{n}(4000)$ spectral index this corresponds to approximate ages of $<0.389$ Gyr and $>{1.586}$ Gyr for young and older populations, respectively. We find that many of our objects fall in a younger zone, i.e., $\text{D}_{n}(4000)<{1.0}$, than the \cite{Kauffmann_2003} range. However, this is an indication that a good number of our dual AGN candidates play host to extremely young stellar populations perhaps of a few $10$ Myr ($0.01$ Gyr). For each of our objects, we also make use of Lick indices [see \cite{Worthey_1997}] obtained from the MPA-JHU DR7 release of spectrum measurements catalogue [see \cite{Kauffmann_2003}; \cite{Brinchmann_2004}; \cite{Salim_2007}], in particular $\text{H}\delta_{A}$.

From \ref{Brinchmann} we find that there are both similarities within our candidate Type II dual AGN sample relative to our control sample and many differences. It is evident that there is a general elevation of star-formation rates for our candidate dual AGN double-peaked Type II AGN sample, as well as a precedent for younger stellar populations. However, the SFR has no real relation to the level of $\text{[O II]}$ relative to the $\text{[O III]}$ flux. This perceived elevation in the $\text{[O II]}$ emission-line may well be due to triggering timings of AGN within merging galaxies. The flux of the $\text{[O III]}$ emission-line scales well with AGN activity; where the peak of AGN accretion should correspond to the peak $\text{[O III]}$ luminosity. The bolometric luminosity, $L_{\text{bol}}$, has also been shown to be in phase with each pericentre passage; triggering at the point of closest SMBH approach and hitting peak luminosity at SMBH separations $<{1-10}$ kpc [\cite{Van_Wassenhove_2012}]. Hence, if the AGN merger is not at this peak triggering phase then the luminosity of the $\text{[O III]}$ emission may well be depressed. If, as is highly plausible, the triggering of peak accretion onto the central SMBH pair is not in phase with the peak of star-formation then observing an elevated $\text{[O II]}$ flux relative to the $\text{[O III]}$ emission may be common for AGN within mergers, and not an indication of the actual current SFR. However, simulations [\cite{Van_Wassenhove_2012}] have shown that the peak SFR phase and peak $L_{\text{bol}}$ are concurrent; somewhat contradicting this notion. What is also noticeable is that the candidate dual Type II AGN represent galaxies with a statistically younger stellar population, with a mean $\text{D}_{n}(4000)$ break index of $1.394$ compared to a mean $\text{D}_{n}(4000)$ break index of $1.467$ for our Type II single AGN control sample. Hence, it is clear than a statistically significant number of candidate dual AGN from our sample have had recent starburst episodes.      

For our candidate Type II dual AGN sample our $\text{[O II]}$ luminosities range from $\sim{8\times{10}^{39}}$ to $4.2\times{10}^{42}$ erg s$^{-1}$, with an average $\text{[O II]}$ luminosity, $\left\langle{L_{\text{[O II]}}}\right\rangle$, of $\sim{3.5\times{10^{41}}}$ erg s$^{-1}$. Applying the maximum plausible AGN contamination extent, as calculated from the statistical analysis of the $\text{[O II]}/\text{[O III]}$ ratio applied to our control sample (as outlined §4.1.1) as well as metallicity and extinction corrections yields an average SFR of $\sim{1.45}M_{\odot}$. We also find that the average $\text{[O III]}$ log  luminosity, $\left\langle\log{L_{\text{[O III]}}}\right\rangle$, of $\sim{41.78}$. As the $\text{[O III]}$ luminosity is often assumed to be a very good proxy indication of AGN power for obscured Type II AGN [\cite{2004ApJ...613..109H}; \cite{Wild_2010}], we note that there is a tendency of the most powerful of the dual AGN candidates to play host to younger stellar populations, where the average $\text{[O III]}$ luminosity for those objects with a measured $\text{D}_n(4000)$ index $<{1.2}$ is of the order of $\sim{10^{42.6}}$, compared with $\sim{10^{41}}$ for object with average older stellar populations (i.e., $\text{D}_n(4000)>{1.5}$). This is consistent with the findings of \cite{Kauffmann_2003} that the most luminous Type II AGN are playing host to an increased $\text{SFR}$. It is evident that a number of dual AGN candidates are not only in a luminous phase of their accretion event, but also undergoing enhanced star-formation; and it may also be suggestive that only the most gas rich and massive galaxy mergers can supply concurrent dual AGN activity with ongoing star-formation. We also note, that although there are major differences between our control pure AGN sample (i.e., AGN with $\text{[O III]}$ $\lambda{5007}$Å$/\text{H}\beta$ > 10), that in comparison with previous work [e.g., \cite{Volonteri_2003}; \cite{Gu_2006}; \cite{Somerville_2008} \cite{Rosario_2012}] our results are consistent with a model where at low-redshifts (i.e., $z<{0.3}$) there is an importance of major-mergers in driving both the growth of SMBHs and host global star-formation of high AGN luminosities ($L_{\text{[O III]}>{10^{42}}}$). We find a few notable objects that are quite luminous in their $\text{[O II]}$ emission, but comparatively much more luminous in their $\text{[O III]}$ emission causing a lower than expected $\text{SFR}$ due to this AGN correction. These objects can be seen as outliers in \ref{Silverman}(b).

Evident in our $\log{L_{\text{[O III]}}}$ vs. $\log{\text{[O II]}/\text{[O III]}}$ plot we see that the most luminous AGN at $\lambda5007$Å (i.e., $\log{\text{[O II]}/\text{[O III]}}>{10^{42}}$) also have much lower $\text{[O II]}/\text{[O III]}$ ratios relative to the whole sample. Interestingly, as mentioned, contradictory to this is that the aggregate $\log{L_{\text{[O III]}}}$ for the most powerful dual AGN are concurrent with hosts that have younger stellar ages. This contradiction is perhaps reconciled by the very nature of the triggering of dual AGN; we expect to see ratios for high star-forming AGN hosts greater than $\sim{0.3}$, whereas for dual AGN the triggering of peak $\text{[O III]}$ luminosity occurs when star-formation is also at its peak. Hence, it seems that there is either a higher than expected contribution to the $\text{[O III]}$ from a star-forming component or the active SMBH luminosity component as a fraction of the $L_{\text{bol}}$ is far more dominant relative to the star-forming component for dual AGN undergoing an active merger. As a further test to this interpretation we also plot $\log{L_{\text{[O III]}}}$ against $\log{L_{\text{[O II]}}}$ (see \ref{Silverman}) and, as a further comparison, the relationship between $\log{L_{\text{[O III]}}}$ and $\text{SFR}_{\text{[O II]}}$.

\subsubsection{SDSS Binary \& Confirmed AGN Sample: Star-Formation \& Stellar Ages}

$\text{SDSS J110851.04+065901.4}$ is a confirmed dual AGN with an $\text{SFR}$ of $\sim{10}$ $M_{\odot}$ yr$^{-1}$), a $\text{D}_{n}(4000)$ spectral index of 1.085 and a $\text{H}\delta_{A}$ Balmer absorption index of 5.237. Such spectral index parameters are archetypal for a merger with recent star-formation occurring within the last $\sim{0.01-0.1}$ Gyr, with a $SFR$ being indicative of a current starburst episode comparable to that of the well-known starburst M82 [\cite{1978ApJ...221...62O}; \cite{1985Ap.....22..142A}].  

\subsubsection{Burst-AGN Timing}
 
\subsubsection{Kolmogorov-Smirnov (K-S) Tests.}

Given the uncertainties and the spread in the value for the major emission-lines $\text{[O II]}$ and $\text{[O III]}$ luminosity, we conducted a non-parametric Kolmogorov-Smirnov (K-S) test [\cite{Peacock_1983}; \cite{Justel_1997}]. This test can be used to examine any potential variability between our dual AGN sample and our control AGN sample as well as to evaluate whether the two Type II narrow-line samples belong to two distinct and different populations (even with a potential large amount of the double-peaked emission line profiles within our candidate dual AGN sample being attributed to outflows, rotating gaseous discs on kpc scales or asymmetrical jets emerging from the central engine). As such, if we assume that such double-peaked emission lines which produce such phenomena can potentially be seen in normal narrow-line Type II AGN we should not see any distinct differences between the two samples, i.e., we should not see an elevated level of star-formation. In essence, the K-S test can help us determine if our two data samples come from the same underlying distribution. We perform the K-S test by using the ad hoc python module, scipy.stats.ks_2samp. We adopt a null hypothesis, $\text{H}_{0}$, whereby the $\text{[O II]}/\text{[O III]}$ two samples are considered to be generically of the same population.  

In \ref{KolmogorovSmirnov} we can see an empirical cumulative distribution function  performed for the $\text{[O II]}/\text{[O III]}$ line ratio for both of our samples as well as a binned distribution histogram for both samples. For our chosen samples, i.e., $n_{1}=n_{\text{AGN}}=100$ and $n_{2}=n_{\text{dAGN}}=148$, the K-S statistic has a value of $0.38$, far greater than the above boundary; we can reject the $\text{H}_{0}$ with a \textit{p-value} of $5.49\times10^{-8}$. This is a clear indication that the two samples came from two different populations. Underlining the diverse nature of the $\text{[O II]}/\text{[O III]}$ line ratio for the two samples, we measure a skewness (i.e., asymmetry of the probability distribution) value of 1.18 for our Type II AGN control sample and a value of 3.54 for our candidate double-peaked Type II dual AGN sample. Essentially, the histogram plot confirms this asymmetry test where the candidate double-peaked Type II dual AGN sample has a higher number of outliers and objects exceeding the $\mu+{2\sigma}$.


  
  
  
  
  
  
  
  
  
  
  
  
  
  
  
  
  
  
  
  
  
  
  
  
  
  
  
  
  
  
  
  