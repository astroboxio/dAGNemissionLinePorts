\section{Results \& Discussions}
\subsection{Sample Population}
\subsection{Ge+2012 Type II Sub-Sample Global SFR}

As alluded to in §3.1.2 an initial proxy of the star-formation rate (SFR) uncorrected for an AGN contribution, metallicity, dust obscuration and reddening was calculated for the entirety of the Ge+2012 DR7 Type II AGN sub-sample using the method outlined in \cite{Kennicutt_1998} and using newly released spectroscopic measurements of the DR7 sample from DR10. From the \cite{Kennicutt_1998} method we obtained a median SFR value of 16.705 M$_{\odot}$ yr$^{-1}$. Utilising the method of \cite{Kewley_2004}, correcting for an AGN contribution advised from \cite{2006ApJ...642..702K} as well as corrections arising from metallicity, dust obscuration and reddening, we further obtained a median AGN correction SFR of 6.221M$_{\odot}$ yr$^{-1}$. We also find from the overall parent sample of DR7 Type II objects that approximately 30\% of objects correspond to SFRs greater than the accepted value for star-forming galaxies of 10M$_{\odot}$ yr$^{-1}$. Of the 30\% that could be classified as star-forming galaxies 35\% have corresponding SFR values greater than 20M$_{\odot}$ yr$^{-1}$, a figure usually associated with starburst galaxies.


