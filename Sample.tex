\section{Sample Population Properties \& Statistics}
\subsection{Population Galaxy Properties}
\subsubsection{SDSS II DR7 MPA-JHU Ge+ dAGN Sample}

The \cite{Ge_Hu_Wang_Bai_Zhang_2012} candidate dAGN population sample from the Sloan Digital Sky Survey (SDSS II) Data Release 7 (DR7) forms the basis of our targetary survey. Their dAGN hunt focused on galaxies observed in the optical wavelength by the Apache Point Observatory 2.5-m wide-angle telescope. Analysis conducted by \cite{Ge_Hu_Wang_Bai_Zhang_2012} conducted on 927,552 galaxy spectra from the SDSS DR7 MPA-JHU5 database\footnote{\tiny{Max-Planck-Institut für Astrophysik/John Hopkins University Data Release 7 (MPA/JHU5 DR7) reference: \tt{http://www.mpa-garching.mpg.de/SDSS/DR7/}.}} SDSS-DR7 [\cite{2009ApJS..182..543A}] using the emission-line measurements of $\text{H}\alpha$, $\text{[O III]}$ and $\text{H}\beta$ found a total of 3030 double-peaked narrow-emission line galaxies (NELs) and 12,582 with asymmetric, top-flat profiles with a $\text{S/N}$ ratios ${>}{9}$ and equivalent widths EW($\text{H}\alpha$, $\text{H}\beta$, or $\text{[O III]}$) ${>}{3}$\AA. The nature of the emission-line profiles were then extracted with STARLIGHT\footnote{\tiny{The SEAGal/STARLIGHT spectral synthesis code: \tt{http://astro.ufsc.br/starlight/}}} with a combination of narrow Gaussian, broad Balmer and $\text{[O III]}$ winged fits. The selection methods for the Ge+2012 sample is outlined in detail in §2 of \cite{Ge_Hu_Wang_Bai_Zhang_2012}. The entirity of the Ge+2012 population sample can be found at: \tt{[http://vizier.cfa.harvard.edu/viz-bin/VizieR?-source=J/ApJS/201/31]}.

The Ge+2012 sample contains a myriad of objects classified as binary objects, for example binary star-forming galaxies, binary Type II AGN and composite or mixed Type II AGN-star forming objects. Of the total initial Ge+2012 sample of 15611 objects, we target our analysis on 3841 objects designated as Ge+2012 Type 2 AGN by the emission-line analysis of the MPA-JHU DR7 ($\sim{0.4}$\% of the total MPA-JHU5 SDSS galaxy sample), i.e., objects which are confirmed binary Type II Seyferts within the low-redshift range (e.g., $0.00824{<}z{<}0.37712$; with galaxy velocity dispersions, $\sigma$ (km s$^{-1}$), ranging from 44.10 to 500). Within this smaller sub-sample of dAGN candidates are other attached peculiarities and emission-line classifications. Associated to the emission-line features are the classifications of `pp', i.e., double-peaked narrow emission-line (NEL) objects, and `as', i.e., assymmetric top-flat NEL objects. Further to the emission-line classifications are the `2` and `A' percularities indicating galaxies with dual-cores and AGNs with strong $\text{[O III]}$ wings, respectively, with a peculiarity designation of A+2 representing objects which show both resolvable dual-cores in SDSS imaging as well as strong $\text{[O III]}$ wings. A full overview of the spectral emission-line features of the Ge+2012 can be found in §2.1, §2.2 and §2.3 of \cite{Ge_Hu_Wang_Bai_Zhang_2012}.   

\subsubsection{SDSS III DR10 Portsmouth Group dAGN Sample}

Further to the DR7 data we analyse the Ge+2012 sample in light of the latest SDSS data release (SDSS III DR10). We cross-reference the Ge+2012 Type 2 Seyfert pair sample with emission-line flux measurements and stellar kinematic measurements [see: \cite{Thomas_2013}] from the Galaxy Properties database from the Portsmouth Group\footnote{\tiny{Galaxy Properties from the Portsmouth Group: \tt{http://www.sdss3.org/dr10/spectro/galaxy_portsmouth.php#kinematics}}}$^{\text{,}}$\footnote{\tiny{Portsmouth stellar kinematics and emission-line flux measurements [Thomas et al. 2013] are based on adaptations of the publicly available codes Penalized PiXel Fitting (pPXF, [Cappellari \& Emsellem 2004]) and Gas and Absorption Line Fitting code (GANDALF v1.5; [Sarzi et al. 2006])}} for SDSS and BOSS galaxies. The Portsmouth Galaxy Group databse utilises GANDALF (\textit{G}as \textit{and} \textit{A}bsorption \textit{L}ine \textit{F}itting code; [\cite{Sarzi_2006}]) and pPXF (\textit{P}enalized \textit{P}i\textit{X}el \textit{F}itting; [\cite{2012ascl.soft10002C}]). This new spectral fitting approach fits stellar population and Gaussian emission line templates to the galaxy spectrum simultaneously to separate stellar continuum and absorption lines from the ionized gas emission using GANDALF. Stellar kinematics are then evaluated by pPXF where the line-of-sight velocity distribution is fitted directly in pixel space. The fits account for the impact of diffuse dust in the galaxy on the spectral shape adopting a \cite{Calzetti_2001} obscuration curve. The code further determines the kinematics of the gas (velocity and velocity dispersion) and measures emission line fluxes and equivalent widths (EWs) on the resulting Gaussian emission line template.

From the initial SDSS DR7 MPA-JHU sample we find 3348 that can be cross-referenced with the \tt{emissionLinesPorts} database from SDSS III DR10 (a total of $\sim{87.16}$\% of the total Ge+2012 sample). This smaller sub-set has had BPT diagram re-classifications: 562 (i.e., $16.78$\%) Seyfert--2, with 2244 composite AGN-SF objects (i.e., $67.02$\%). The remaining are re-classified as either LINER or star-forming galaxies. Further to the designations from the \tt{emissionLinesPorts} DR10 database we analyse the redshifted and blueshifted components in light of the new SDSS III analyses.