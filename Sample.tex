\section{Sample Population Properties \& Statistics}
\subsection{Population Galaxy Properties}
\subsubsection{SDSS II DR7 MPA-JHU Ge+ dAGN Sample}

Over the past decade, since the inception of the Sloan Digital Sky Survey (SDSS), the ability to study the host galaxies of AGN has vastly increased. An offshoot of the many numerous parent AGN samples [e.g., \cite{Zakamska2003}; \cite{Kewley_2006}; \cite{Stasinska_2006}; \cite{Wild_2010}] are the dual AGN candidate samples [\cite{Smith_2010}; \cite{Liu_2012}; \cite{Ge_Hu_Wang_Bai_Zhang_2012}; \cite{2012ApJ...753...42C}]. Both the \cite{Ge_Hu_Wang_Bai_Zhang_2012} double-peaked [O III] profile candidate dAGN sample and the \cite{Liu_2012} sample of AGN pairs form the basis of our exploratory study. 

Analysis conducted by \cite{Ge_Hu_Wang_Bai_Zhang_2012} conducted on 927,552 galaxy spectra from the SDSS DR7 MPA-JHU5 database\footnote{\tiny{Max-Planck-Institut für Astrophysik/John Hopkins University Data Release 7 (MPA/JHU5 DR7) reference: \tt{http://www.mpa-garching.mpg.de/SDSS/DR7/}.}} SDSS-DR7 [\cite{2009ApJS..182..543A}] using the emission-line measurements of $\text{H}\alpha$, $\text{[O III]}$ and $\text{H}\beta$ found a total of 3030 double-peaked narrow-emission line galaxies (NELs) and 12,582 with asymmetric, top-flat profiles with a $\text{S/N}$ ratios ${>}{9}$ and equivalent widths EW($\text{H}\alpha$, $\text{H}\beta$, or $\text{[O III]}$) ${>}{3}Å$. The nature of the emission-line profiles were then extracted with STARLIGHT\footnote{\tiny{The SEAGal/STARLIGHT spectral synthesis code: \tt{http://astro.ufsc.br/starlight/}}} with a combination of narrow Gaussian, broad Balmer and $\text{[O III]}$ winged fits. The selection methods for the Ge+2012 sample is outlined in detail in §2 of \cite{Ge_Hu_Wang_Bai_Zhang_2012}. The entirety of the Ge+2012 double-peaked and asymmetric NEL population sample can be found at: \tt{[http://vizier.cfa.harvard.edu/viz-bin/VizieR?-source=J/ApJS/201/31]}.

Whilst the \cite{Ge_Hu_Wang_Bai_Zhang_2012} sample is numerous in its extent, a number of objects still hold ambiguities in their classification (e.g., prescribed therein as 'Type 5') as well as a number of objects are also designated as Type 1, that is they are double-peaked broad-line Type I AGN and their dual nature was not confirmed conclusively. Conversely, the \cite{Liu_2012} sample of Type II (narrow-line) AGN and composite pairs provides us with an interesting cross-reference for the \cite{Ge_Hu_Wang_Bai_Zhang_2012} sample; providing our original Ge+2012 sample with confirmed AGN pairs with tidal classifications (i.e., 0 = non-interactions; 1 = ambiguous; 2 = tidal; 3 = 'dumbbell systems'), transverse proper separations, $\text{r}_{\text{p}}$ ($\text{h}_{70}^{-1}$ $\text{kpc}$), and angular separations, $\Delta\Theta$. Focusing on SDSS Type II narrow-line AGN as designated by \cite{Liu_2012} will allow much more accurate measurements of the SFR as the optical light is not strongly influenced by the activity of the SMBH nucleus in narrow-line AGN.

From the original Liu+2011 sample of 2818, as stated above we focused our attention on Type II narrow-line AGN pairs (a total of 2532) and looked to cross-reference these objects with the full Ge+2012 sample (a total of 15612). From this cross-referencing we have a total of 99 objects which are designated as narrow-line Type II AGN or NL quasar (QSO) pairs by Lui+2011 (QSO pairs attributed to Reyes et al. 2008) and are also designated as either Type II + Type II, Type II + SF or ambiguous/unclassified by Ge+2012. This cross-correlation not only allows us to understand the emission profile and fluxes of the redshifted and blueshifted [O III] components, but also allows us to understand the interaction or merger-phase of the host AGN pairs. Ideally, the cross-comparative study will allow us to identify SFR for individual systems whilst simultaneously having an understanding of the transverse proper separation, $\text{r}_{\text{p}}$, and tidal features giving a more accurate picture of the SFR as related to the merger stage, i.e., close-interaction, distant-interaction, etc. Furthermore, we can fully utilise the flux measurements of these objects from DR10 from the SDSS as well as making further comparisons with local narrow-line AGN in the same redshift extent as our parent sample. The entirety of the Liu+2011 AGN pair population sample can be found at: \tt{[http://vizier.cfa.harvard.edu/viz-bin/VizieR?-source=J/ApJ/737/101]}.

The Ge+2012 sample contains a myriad of objects classified as binary objects, for example binary star-forming galaxies, binary Type II AGN and composite or mixed Type II AGN-star forming objects. Of the total initial Ge+2012 sample of 15611 objects, we target our analysis on 3841 objects designated as Ge+2012 Type 2 AGN by the emission-line analysis of the MPA-JHU DR7 ($\sim{0.4}$\% of the total MPA-JHU5 SDSS galaxy sample), i.e., objects which are confirmed binary Type II Seyferts within the low-redshift range (e.g., ${0.00824}<{z}<{0.37712}$; with galaxy velocity dispersions, $\sigma$ (km s$^{-1}$), ranging from 44.10 to 500). The motivations behind this choice coincides with the Ge+2012 catalogue containing a large number of prospective Seyfert Type II binary AGN as well as the prevalence of young stellar populations within the bulges of nearby Seyfert galaxies [e.g., \cite{1990MNRAS.242..271T}; \cite{Delgado_2001}] and a general global enhanced star-formation prevalence within Seyfert IIs [e.g., \cite{2006MNRAS.366..480G}].  

Within this smaller binary Type II sub-sample of dAGN candidates are other attached peculiarities and emission-line classifications. Associated to the emission-line features are the classifications of `pp', i.e., double-peaked narrow emission-line (NEL) objects, and `as', i.e., asymmetric top-flat NEL objects. Further to the emission-line classifications are the `2` and `A' peculiarities indicating galaxies with dual-cores and AGNs with strong $\text{[O III]}$ wings, respectively, with a peculiarity designation of A+2 representing objects which show both resolvable dual-cores in SDSS imaging as well as strong $\text{[O III]}$ wings. A full overview of the spectral emission-line features of the Ge+2012 can be found in §2.1, §2.2 and §2.3 of \cite{Ge_Hu_Wang_Bai_Zhang_2012}. Overall, we should combine the strengths of the two samples to give a much broader picture of the dual AGN star formation connection.    

\subsubsection{SDSS III DR10 Portsmouth Group dAGN Sample}

Further to the DR7 data we analyse the Ge+2012 sample in light of the latest SDSS data release (SDSS III DR10). We cross-reference the Ge+2012 Type 2 Seyfert pair sample with emission-line flux measurements and stellar kinematic measurements [see: \cite{Thomas_2013}] from the Galaxy Properties database from the Portsmouth Group\footnote{\tiny{Galaxy Properties from the Portsmouth Group: \tt{http://www.sdss3.org/dr10/spectro/galaxy_portsmouth.php#kinematics}}}$^{\text{,}}$\footnote{\tiny{Portsmouth stellar kinematics and emission-line flux measurements [Thomas et al. 2013] are based on adaptations of the publicly available codes Penalized PiXel Fitting (pPXF, [Cappellari \& Emsellem 2004]) and Gas and Absorption Line Fitting code (GANDALF v1.5; [Sarzi et al. 2006])}} for SDSS and BOSS galaxies. The Portsmouth Galaxy Group database utilises GANDALF (\textit{G}as \textit{and} \textit{A}bsorption \textit{L}ine \textit{F}itting code; [\cite{Sarzi_2006}]) and pPXF (\textit{P}enalized \textit{P}i\textit{X}el \textit{F}itting; [\cite{2012ascl.soft10002C}]). This new spectral fitting approach fits stellar population and Gaussian emission line templates to the galaxy spectrum simultaneously to separate stellar continuum and absorption lines from the ionized gas emission using GANDALF. Stellar kinematics are then evaluated by pPXF where the line-of-sight velocity distribution is fitted directly in pixel space. The fits account for the impact of diffuse dust in the galaxy on the spectral shape adopting a \cite{Calzetti_2001} obscuration curve. The code further determines the kinematics of the gas (velocity and velocity dispersion) and measures emission line fluxes and equivalent widths (EWs) on the resulting Gaussian emission line template.

From the initial SDSS DR7 MPA-JHU sample we find 3348 that can be cross-referenced with the \tt{emissionLinesPorts} database from SDSS III DR10 (a total of $\sim{87.16}$\% of the total Ge+2012 sample). This smaller sub-set has had BPT diagram re-classifications: 562 (i.e., $16.78$\%) Seyfert--2, with 2244 composite AGN-SF objects (i.e., $67.02$\%). The remaining are re-classified as either LINER or star-forming galaxies. Further to the designations from the \tt{emissionLinesPorts} DR10 database we analyse the redshifted and blueshifted components in light of the new SDSS III analyses.

\subsection{Control Samples}

\subsubsection{SDSS AGN Control Sample}

To constrain the AGN contribution to the $\text{[O II]}$ emission, we again utilised the vast spectroscopic database of the SDSS in the search for narrow-line Type II AGN. To confirm an object as a Type II AGN for this study we impose the following criteria: (1) the emission-line ratio $\log(\text{[O III]}/\text{H}\beta)>{0.690}$, with a preference for $\text{[O III]}$ $\lambda$5007/$\text{H}\beta$ > 10; (2) the flux profiles are best fitted with Gaussian components with velocity dispersions of 600 < FWHM (km s$^{-1}$) < 1000; (3) a S/N ratio > 8, with an ideal preference towards high S/N objects; (4) the maximum redshift is $\sim{0.3}$ to allow for the $\text{[N II]}$ $\lambda$6584/$\text{H}\alpha$ complex to not be excluded from the spectral observation. As with our dual AGN sample, importance is placed on on the Balmer decrement $\text{H}\alpha$/$\text{H}\beta$ measurement. In the cases where $\text{H}\beta$ is measurable, the width of the $\text{H}\beta$ component was fixed to the best-fit value of the $\text{[O II]}$ component.

\subsubsection{Darg Galaxy Zoo Mergers Control Sample}

The next control sample was based on the Galaxy Zoo mergers catalogue [\cite{2010MNRAS.401.1552D}] was produced via the Galaxy Zoo (GZ) project [\cite{2008MNRAS.389.1179L}]. GZ is uniquely powerful in detecting rare classes of objects like mergers, which can only be reliably selected via direct visual inspection of galaxy images. GZ had enlisted over 500,000 volunteers from the general public to morphologically classify (as spiral, elliptical and mergers), through visual inspection, the entire SDSS spectroscopic sample [York et al. 2000; \cite{2008ApJS..175..297A}]. This preliminary classification was used by \cite{2010MNRAS.401.1043D} to build a merger catalogue of over three thousand objects. The raw parameter $f_{\text{m}}$, called the \textit{weighted merger-vote} fraction, is used to select the correct galaxies from the whole sample which quantifies the probability that a certain image was in fact the image of a merger based on all the classifications of that particular image. The parameter $f_{\text{m}}$ ranges from $0$ to $1$ so that an object with $f_{\text{m}}=0$ should look nothing like a merger and $f_{\text{m}}=1$ should look unmistakably so. To find mergers within the GZ catalogue, a cut of ${0.4}<{f_{\text{m}}}<{1.0}$ was applied. The reason behind the cut on the weighted merger-vote fraction was the high occurrence of false positives (which are virtually non-existent in the interval $f_{\text{m}}>{0.6}$). A second layer of visual inspection, performed by the team, was used to remove any non-merging systems, visually select an appropriate SDSS object to represent the merging partner, and assign morphologies to the galaxies in each merging system. The final product was the compilation of the largest, most homogeneous sample of merging systems in the local Universe to date [\cite{2010MNRAS.401.1552D}; \cite{2010MNRAS.401.1043D}]. This catalogue was uniquely ideal to use as a comparison with our own sample, since the Darg catalogue is \textit{AGN-blind}.

  
  
  
  