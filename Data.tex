\section{Data}
\subsection{Parent Sample}

\subsubsection{SDSS II DR7 MPA-JHU Type II dual AGN Sample: Liu+2010}

In order to better understand the star-formation properties of dual AGN we started with a sample of 167 Type II dual AGN from \cite{Liu_2009}. The objects within this candidate dual AGN sample where selected on the following criteria: (1) the rest-frame wavelength ranges [4700, 5100]\AA{} and [4982, 5035]\AA{} centred on the $\text{[O III]}$ $\lambda$$5007$\AA{} line have median signal-to-noise ratio (S/N) > 5 pixel$^{−1}$ and bad pixel fraction <30\%; (2) the $\text{[O III]}$ $\lambda$$5007$\AA{} line is detected at >5$\sigma$ and has a rest-frame EW > 4\AA; (3) the line flux ratio $\text{[O III]}$ $\lambda$$5007$/$\text{H}\beta$ > 3 if ${z}<{0.33}$, or the diagnostic line ratios $\text{[O III]}$ $\lambda$$5007$/$\text{H}\beta$ and $\text{[N II]}$ $\lambda$$6584$/$\text{H}\alpha$ lie above the theoretical upper limits for star-formation excitation from \cite{Kewley_Dopita_Sutherland_Heisler_Trevena_2001} on the BPT emission-line diagnostic diagram [\cite{Baldwin_1981}] if ${z}<{0.33}$. From these constraints \cite{Liu_2009} selected a total of 167 objects which yielded unambiguous double-peaked $\text{[O III]}$ emission lines. The Liu+2010 candidate dual AGN sample has a redshift extent of ${0.0319}<{z}<{0.6427}$, hence some objects will have their $\text{[N II]}$ $\lambda$$6584$/$\text{H}\alpha$ complex excluded from the spectral observation. For our analysis we exclude those objects where their redshifts do not allow for the $\text{[N II]}$ $\lambda$$6584$/$\text{H}\alpha$ complex to be visible in the optical window. Importance is also placed on the distinguishability of the $\text{H}\beta$ emission profile as we place an importance on the Balmer decrement $\text{H}\alpha/\text{H}\beta$ of both our dual AGN sample and our local AGN control sample.

As outlined in \cite{Liu_2009}, the Type II candidate dual AGN sample shows systematically larger stellar velocity dispersions and masses, older mean stellar ages (and redder colours) and smaller fractions of post-starburst populations in the past $0.1–1$ Gyr compared to their AGN control sample. This somewhat contradicts the idea that dual AGN should host higher levels of star-formation than their single AGN counterparts if they are indeed a result of two merging galaxies. However, as discussed in \cite{Liu_2009}, and others, many dual AGN may be attributed to narrow-line region kinematics (such as rotating gaseous discs on kpc scales, asymmetrical jets or biconical outflows). Hence, within our parent candidate dual AGN sample there is a high certainty that not all double-peaked emission line objects are attributable to dual AGN; however, \cite{Liu_2009} state that the double-peaked objects are not exclusively attributed NLR kinematics due to a lack of ionisation stratification, with consistent velocity offsets in $\text{H}\beta$ and $\text{[O III]}$ also being measured.

\subsection{Control Sample}

\subsubsection{SDSS AGN Control Sample}

To constrain the AGN contribution to the $\text{[O II]}$ emission, we again utilised the vast spectroscopic database of the SDSS in the search for narrow-line Type II AGN. To confirm an object as a Type II AGN for this study we impose the following criteria: (1) the emission-line ratio $\log(\text{[O III]}/\text{H}\beta)>{0.690}$, with a preference for $\text{[O III]}$ $\lambda$5007/$\text{H}\beta$ > 10; (2) empirical distinctions between Seyfert AGN and low-ionization nuclear emission-line region galaxies (i.e., LINERs) whereby the demarcation ${1.89\log{(\text{[S II]}/\text{H}\alpha)}+0.76}<{\log{(\text{[O III]}/\text{H}\beta)}}$ is held (3) the flux profiles are best fitted with Gaussian components with velocity dispersions of 600 < FWHM (km s$^{-1}$) < 1000 attributable to narrow-line regions; (4) a S/N ratio > 8, with an ideal preference towards high S/N objects; (5) the maximum redshift is $\sim{0.3}$ to allow for the $\text{[N II]}$ $\lambda$6584/$\text{H}\alpha$ complex to not be excluded from the spectral observation. As with our dual AGN sample, importance is placed on on the Balmer decrement $\text{H}\alpha$/$\text{H}\beta$ measurement. In the cases where $\text{H}\beta$ is measurable, the width of the $\text{H}\beta$ component was fixed to the best-fit value of the $\text{[O II]}$ component. Focusing on SDSS Type II narrow-line AGN will allow much more accurate measurements of the SFR as the optical light, particularly the aforementioned $\text{[O II]}$ doublet and the $\text{H}\alpha$ emission-lines, is not strongly influenced by the activity of the SMBH nucleus in narrow-line AGN.

\subsection{Sample Measurements: Optical SDSS Spectra}

\subsubsection{Emission Line Flux Measurements, Stellar Continuum Extraction \& Stellar Kinematics}

Utilising the open access SDSS spectra, we performed measurements of the integrated flux (in units of $10^{-17}$ erg s$^{-1}$ cm$^{-2}$) of the $\text{[O II]}$ $\lambda\lambda$$3727.09$, $3729.88$\AA\AA{} doublet as well as the double-peaked emissions from the $\text{H}\beta$, $\text{[O III]}$ $\lambda$$5008.24$\AA{}, $\text{H}\alpha$ and $\text{[N II]}$ $\lambda$$6585.27$\AA{} lines, host stellar velocity dispersion, the velocity offsets for the blueshifted and redshifted components to the $\text{H}\beta$ and $\text{[O III]}$ $\lambda$$5008.24$\AA{} emission-lines for our double-peaked emission lines in Type II AGN sample. Likewise, we also perform the same spectral flux measurements on the emission-lines of our pure Seyfert-II AGN sample, whilst also obtaining r band magnitudes for all of our objects in both the parent and control samples from the SDSS MPA-JHU database. 

Measurements of the stellar kinematics, velocity and velocity dispersions were obtained by utilising the Portsmouth Galaxy group method\footnote{\tiny{Galaxy Properties from the Portsmouth Group: \tt{http://www.sdss3.org/dr10/spectro/galaxy_portsmouth.php#kinematics}}}$^{\text{,}}$\footnote{\tiny{Portsmouth stellar kinematics and emission-line flux measurements [Thomas et al. 2013] are based on adaptations of the publicly available codes Penalized PiXel Fitting (pPXF, [Cappellari \& Emsellem 2004]) and Gas and Absorption Line Fitting code (GANDALF v1.5; [Sarzi et al. 2006])}} for SDSS and BOSS galaxies. This method utilises GANDALF (\textit{G}as \textit{and} \textit{A}bsorption \textit{L}ine \textit{F}itting code; [\cite{Sarzi_2006}]) and pPXF (\textit{P}enalized \textit{P}i\textit{X}el \textit{F}itting; [\cite{2012ascl.soft10002C}]). This new spectral fitting approach fits stellar population and Gaussian emission line templates to the galaxy spectrum simultaneously to separate stellar continuum and absorption lines from the ionized gas emission using GANDALF. Stellar kinematics are then evaluated using pPXF where the line-of-sight velocity distribution is fitted directly in pixel space. The fits account for the impact of diffuse dust in the galaxy on the spectral shape adopting a \cite{Calzetti_2001} obscuration curve. The code further determines the kinematics of the gas (velocity and velocity dispersion) and measures emission line fluxes and equivalent widths (EWs) on the resulting Gaussian emission line template. 

After subtracting the continuum models, we utilise the open access python based spectral fitting routine (\tt{pyspeckits}; \cite{2011ascl.soft09001G}) to fit two Voigt profiles to the double-peaks of the above mentioned emission-lines with half widths of $\sigma\sim{1.53}$ and ${0.00}<{\gamma}<{1.00}$ to obtain the best goodness of fit parameter for each profile (i.e., a reduced $\chi^{2}$ statistic $\sim{1}$). The $\text{[O II]}$ $\lambda\lambda$$3727.09$, $3729.88$\AA\AA{} doublet was treated separately to the ionisation ratio emission-lines, performing a normal Voigt fit centred on both the doublet theoretical peaks. Utilising the redshift values quoted by the SDSS we converted to the luminosity distance using the cosmological parameters quoted in §1.1 obtaining the $\text{[O III]}$ luminosity accordingly.
  
  
  