\label{Zakamska} \textbf{Figure} \ref{Zakamska}. Emission-line properties of dual AGN. \textit{Top left}: plot of $\log{(L_{\text{[O III]}})}$ vs. $\log{(L_{\text{[O II]}})}$ for our candidate dual AGN sample, for objects described as composite objects by the Kewley-Kauffmann criteria (dark purple), as Type II AGN (with $\log(\text{[O III]}/\text{H}\beta)>{0.490}$ and $\log(\text{[N II]}/\text{H}\alpha)>{-0.220}$; black) as well as weak (light grey) and strong star-forming (as given by the Stravinska et al. 2011 demarcation; dark grey). \textit{Bottom left}: plot of $\log{(L_{\text{[O III]}})}$ vs. $\log{(\text{[O II]/[O III]})}$ for our candidate dual AGN sample; with the same descriptive elements as above. Both plots are underlayed with our control sample from the SDSS galaxy merger catalogue as described in Darg et al. (2010) as a comparative illustration. The dotted-dashed horizontal line denotes the \cite{Ferland_1986} $\text{[O II]/[O III]}$ ratio of $\sim{0.1-0.3}$ (i.e., using $\log{\text{[O II]/[O III]}}=-0.52287$) as well as the lower bound by using the hatched region. We also donate the \cite{Zakamska2003} median value for Type II Quasars, of $\log{\text{[O II]/[O III]}}=-0.12493$, denoted by the dotted line.
  
  