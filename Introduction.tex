\section{Introduction}

Discerning star-formation from within AGN host galaxies has often been a divisively complex issue within the field of galaxy evolution. Many authors subscribe to the notion that star-formation, and by association starburst galaxies, is best driven by gas-rich spiral-spiral mergers (major mergers) between galaxies [e.g., \cite{thews_Neugebauer_Scoville_1988}; \cite{Mihos_Hernquist_1996}; \cite{1997AA...326..537D}; \cite{Barton_Geller_Kenyon_2000}; \cite{Tissera_Dominguez-Tenreiro_Scannapieco_Saiz_2002}; \cite{2005A&A...438...87K}; \cite{Narayanan_Groppi_Kulesa_Walker_2005}; \cite{Narayanan_Hayward_Murray_2013}; \cite{Scott_Kaviraj_2013}]. A correlated phenomenon coupled to the interaction and/or merger of gas rich galaxies are dual active galactic nuclei (dAGN). Both numerical simulation [e.g., \cite{Hoffman_Loeb_2007}; \cite{a_Eichhorn_Makino_Spurzem_2010}; \cite{Shen_2010}; \cite{Yu_2011}; \cite{2013MNRAS.429.2594B}; \cite{Kulkarni_Loeb_2012}; \cite{Van_Wassenhove_2014}] and observation [e.g., \cite{ice_Ivison_Lin_Koo_et_al__2007}; \cite{2009ApJ...702L..82C}; \cite{ario_Shields_Smith_Wright_2011}; \cite{nen_Chapman_Bonning_Chiba_2012}; \cite{Barrows_2013}; \cite{Huang_2014}] provide strong evidence in support of the dual active galaxy (dAGN) paradigm, where galaxy mergers produce multiple massive black holes at the centre of the merger remnant which ultimately end up accreting on kpc-scale separations [e.g., \cite{Liu_Greene_Shen_Strauss_2010}; \cite{obel_Comerford_Middelberg_2014}].  Within this `evolution-through-merger' paradigm exists the tenant that mergers and interactions of galaxies play a crucial role in the onset of intense star formation phenomena in galaxies; specifically starbursts [\cite{1984MNRAS.209..111J}]. While seemingly undisturbed AGN are observed, high resolution imaging of quasars (QSOs) show close companions with features of gravitational disturbances due to a recent interaction in a statistically reliable number of cases [e.g., \cite{Bahcall_Kirhakos_Saxe_Schneider_1997}; \cite{Bennert_2008}]. That is to say, a high number of active galaxies show evidence of tidal disruption synonymous with a recent or past merging event [e.g., \cite{Sanders_1988}; \cite{Hutchings_1988}; \cite{Graham_1990}; \cite{Keel_1996}; \cite{Schmitt_2001}; \cite{Alonso_2007}]. One could easily theorise that dual active galaxies may also contain intense global star formation, under the proviso that the merging galaxies play host to enough gas to fuel both phenomena simultaneously.  

Over the past decade, the evidence for the connection between active galactic nuclei (AGN) and enhanced star formation has increased where observations have shown that AGN have a higher propensity to be found within a high-SFR host [e.g., \cite{2001ApJ...555..719C}; \cite{2004MNRAS.352..399J}; \cite{Sanchez_2004}; \cite{2006AJ....131...84V}; \cite{2006AJZakamska1496Z}; \cite{2011MNRASKaviraj}]. \cite{2011ApJ...743....2S} also speculate that close kinematic pairs (merging galaxies) are conducive environments for black hole growth, indicating a causal physical connection to enhanced star formation and AGN within the galaxy merger paradigm. Therefore, if we can conclude with some certainty that dAGN or binary AGN are likely to be a result of merging galaxies then we should see an associated enhanced SFR within dAGN candidate samples.

Numerical simulations conducted by \cite{Dotti_Bellovary_Callegari_2012} show that synchronous dAGN-starburst phases (with bolometric luminosities $>10^{44}\text{erg}\text{s}^{-1}$ and $\text{SFRs}>$10M$_{\odot}$ yr$^{-1}$) are plausible and are likely to be observed during the latter stages of two merging galaxies: i.e., the peak of the merger coincides with the peak of AGN and star formation activity. Such plausibility in observations would, on face value, yield vast numbers of such objects. However, as yet, no major study has focused on such a classification. The \cite{Dotti_Bellovary_Callegari_2012} simulations show that simultaneous dAGN and starburst triggering requires gas rich, major mergers with spiral (S0, Sa-c, SBa-Bc) Hubble morphological classifications. This is in agreement with \cite{Koss_2012}'s study of 168 Swift BAT and SDSS AGN who found that dAGN are far less likely to be found in elliptical morphologies, where minor mergers (i.e., $M_{1}/M_{2}>3$) are $\sim{3}$ times less likely to host high SFRs than mergers with mass ratios of $M_{1}/M_{2}={1-2}$). Bringing all of the associated evidence together, dAGN may provide a unique opportunity to begin to test how tentative the link is between AGN, major mergers and starburst activity in the local Universe (i.e., $z<{0.3}$).             

Another difficulty in attempting to understand the evolutionary history of dAGN is the relatively low propensity of such objects in large scale observations. Observations have found that dAGN fractions in sample AGN data are extremely low, given the large number of galaxy interactions and mergers. One plausible explanation for the lack of dAGN propensity may lie in understanding the link between starburst galaxies, obscuration of the AGN central engine and scarcity of gas for the central engine due to the rapid star-forming phase of a galactic merger: `gas aridity'. \cite{tini_Bongiorno_Piconcelli_2013} showed that a large fraction ($\sim{80}$\%) of luminous AGNs ($L_X$ $\ge$ 10$^{44.5}$ erg s$^{-1}$) should be in starburst galaxies. At present, observations indicate that at least $\simeq{50}$\% of QSOs at redshifts $2.0<z<6.5$ play host to intense star formation. \cite{tini_Bongiorno_Piconcelli_2013} also find that obscuration hampers the observability of luminous AGN in gas rich galaxies with high relative star formation rates, local-SFRs): i.e., the probability of finding a luminous AGN in a gas rich galaxy is low. Somewhat important to understanding the relatively low propensity of observed dAGN in the local Universe is the extent that obscuration of the central nuclear region effects potential observations of the central engines within a galaxy merger: where the gas content is assumed to be high [\cite{Levine_Gnedin_Hamilton_2010}; \cite{Dotti_Bellovary_Callegari_2012}].

The difficulties that one might find in attempting to link the two associated phenomena is a diversely complex issue. Firstly, for dAGN and starburst phases to coincide simultaneously there is a requirement that the galaxy is more than likely going to be gas rich [\cite{Hopkins_2012}]. Although this is good for fuelling several phenomena at once, the relativity propensity of gas may cause the accretion rate of the two central SMBHs to increase, resulting in a much shorter window of observation, i.e., perhaps no greater than a few hundred Myr. There is also increasing evidence for a delay between the peak of a starburst phase and that of the peak accretion onto the SMBH [e.g., \cite{Di_Matteo_2005}; \cite{Wild_2010}; \cite{Hopkins_2012}; Blank \& Duschl (2014)] (i.e., the AGN phase). Coupled to this, evidence has also  suggested that accretion rates for dual central AGN pairs have been found to be inefficient across multiple SMBHs in comparison to single AGN SMBH [\cite{Dotti_Bellovary_Callegari_2012}]. \cite{Dotti_Bellovary_Callegari_2012} predict that for 10\% of the merger timescale much of the AGN activity during the merger is non-simultaneous  Larger-mass SMBHs generally show higher mass accretion rates when normalised to SMBH mass. Evidence suggests that non-synchronous mass accretion onto SMBHs in gas-- and dust--rich IR luminous merging galaxies hampers the observational detection of kpc--scale multiple active SMBHs. Hence, asynchronous accretion may provide an even slighter window of observational opportunity (not only for dAGN-starbursts but also for regular dAGN as a galaxy sub-type. 

Finally, the relatively low propensity of dAGN ($\sim{0.5}$\%--$2.5$\% [\cite{Shen_Liu_Greene_Strauss_2011}]; $1.78$\% [\cite{Ge_Hu_Wang_Bai_Zhang_2012}]; $10.45{\pm}7.55$\% [\cite{rford_Schluns_Greene_Cool_2013}]; $13.8$\% [\cite{2014ApJ...780..106I}]) provides further observational difficulties in searching for simultaneous dAGN-starburst galaxies. dAGN are almost exclusively assumed to form through major mergers, with the possibility of minor mergers and interactions being far more unlikely. However, major mergers between galaxies give rise to a few issues for dAGN formation. For example, major multiple-mergers may cause recoiling SMBH pairs if the differences in central SMBH masses are of the order $\sim{10}$ [\cite{onning_Shields_Salviander_2007}; \cite{Gualandris_Merritt_2008}; \cite{Blecha_Civano_Elvis_Loeb_2013}]. Such mergers may result in very large separations $\sim{600}$ kpc where the recoiling black hole loses its accretion disc and subsequently becomes inactive. Also, differences in the gas properties of galaxies as a function of redshift also plays a major role: e.g. high redshift galaxies are more gas rich and therefore able to achieve high star formation rates through secular evolution [\cite{Hicks_Friedrich_Sternberg_2007}; \cite{2010ApJ...715..202H}; \cite{Draper_Ballantyne_2012}]. Thus, perhaps making major mergers more likely to trigger very highly obscured central AGN [\cite{2009ApJ...698L.116P}] or possibly, and rather disappointingly, halting AGN activity at the close pair stage altogether [\cite{Scott_Kaviraj_2013}].

It has also been noted that star formation found in gas-rich systems suggests that the star formation efficiency in the host galaxies is suppressed in the presence of strong AGN feedback [\cite{1999ApSS.266..207M}; \cite{2004AAS...204.4905B}; \cite{Di_Matteo_2005}]. In contradiction to this, the finding that approximately 30-50\% of Seyfert 2 nuclei are associated with young stellar populations [e.g. \cite{Delgado_2001}; \cite{Storchi_Bergmann_2001}; \cite{Sarzi_2006}] seems to suggest that AGN activity enhances star-formation, and/or that circumnuclear star-formation may enhance AGN activity. \cite{Silverman_2009} found that the vast majority of AGN hosts at $z<1$ have star formation rates (SFRs) that are higher than that of non-AGN hosts of similar mass, and that the incidence of AGN activity increases with decreasing stellar age: suggesting a tentative link between AGN and mergers. So the question therefore remains on many topics relating to merging and active galaxies.

Over the past decade, since the inception of the Sloan Digital Sky Survey (SDSS), the ability to study the host galaxies of AGN has vastly increased. An offshoot of the many numerous parent AGN samples [e.g., \cite{Zakamska2003}; \cite{Kewley_2006}; \cite{Stasinska_2006}; \cite{Wild_2010}] are the dual AGN candidate samples [\cite{Ge_Hu_Wang_Bai_Zhang_2012}; \cite{2012ApJ...753...42C}]. 

We aim to understand the relationship between Seyfert II pair dAGN selected from the Sloan Digital Sky Survey (SDSS) DR7\footnote{\tiny{SDSS SkyServer DR7 reference: \tt{http://cas.sdss.org/astrodr7/en/}}} [\cite{2009ApJS..182..543A}] and their associated measured star formation rates. The catalogue provided in \cite{Ge_Hu_Wang_Bai_Zhang_2012}\footnote{\tiny{VizieR Centre de Donn\'{e}es astronomiques de Strasbourg (CDS) database reference; `Emission-line galaxies from SDSS. I. (Ge+, 2012) survey': \tt{vizier.cfa.harvard.edu/.../VizieR?-source=J/ApJS/201/31}.}} provides extensive pp-AGN candidates. This catalogue formed the statistical and targetary basis of our study. This catologue, herein refered to as Ge+2012, contain both asymmetric and double-peaked narrow emission line (NEL) Seyfert II galaxies where the $\text{H}\alpha$, $\text{[O III]}$ $\lambda{5007}$ and $\text{H}\beta$ lines have equivalent widths greater than 3\AA{ }($\sim{0.3}$ nm). Supplemented to the Ge+2012 spectral emission analysis of SDSS DR7 galaxies, we provide insight into the star-formation rates of the Ge+2012 Seyfert II sample using the analysis conducted by the Portsmouth Group\footnote{\tiny{Galaxy Properties from the Portsmouth Group reference: \tt{https://www.sdss3.org/dr10/spectro/galaxy_portsmouth.php}.}}. The aim of this paper is to understand the relationship and propensity between global star-formation activity and mergers within the dual AGN paradigm. Throughout this paper, we assume a flat ($K=0$) $\Lambda$-CDM standard cosmology with $\text{H}_0={70.4}^{+1.3}_{-1.4}$km s$^{−1}$ Mpc$^{−1}$, $k={0.002}$, $w$=$-0.98\pm{0.053}$, ${\Omega_{m}}={0.27}$ and ${\Omega_{\Lambda}}={{0.728}^{+0.015}_{-0.016}}$ [\cite{Spergel_2003}; \cite{Reid_2010}; \cite{Jarosik_2011}] for all luminosity distance calculations, $\text{D}_L$ (Mpc).