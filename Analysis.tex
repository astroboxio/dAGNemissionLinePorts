\section{Methods, Analysis \& Discussions}
\subsection{Constraining dAGN Star-Formation Rates}
\subsubsection{Constraining AGN Contributions}

As we have outlined, one of the purposes of this paper is to understand the contribution from our sample of dual-AGN to the $\text{[O II]}$ $\lambda{3727}Å$ (i.e., the $\lambda\lambda{3727}$,$3729ÅÅ$ doublet). Hence, a considered evaluation of the AGN contribution to the [O II] emission line is required even though the production of this low ionization line has been shown to be relatively weak in AGN [e.g., \cite{Ferland_1986}; \cite{Ho_1993}; \cite{2006ApJ...642..702K}]. \cite{2009ApJ...696..396S} also showed that the host galaxies of AGN have ongoing star-formation with a broad range of rates ($\approx{1-100}$M$_{\odot}$ yr$^{-1}$) higher than that of the overall massive ($\log{M}>{10.6}$) galaxy population and essentially equivalent to those forming stars (i.e., emission-line galaxies).

\subsubsection{${\text{[O II]}}$ SFR Indicator \& AGN Contribution Corrections (Pure Seyferts)}

Determining the star-formation rates (SFRs) of dual AGN host galaxies is a diveresely complex issue. Our primary obstacle is that when analysing the SFR of an AGN one must account for contributions to the flux of our chosen SFR tracer from both the star-forming regions and the AGN itself. Added to this, the majority of objects within the chosen sample are potentially in mid-collision; or somewhere later in their evolution (e.g., ${\tau}>{1.0}$ Gyr), perhaps nearer to the coalescence stage of their SMBH binary pairs ($\lesssim{10}$ kpc). As such, we expect emissitivity and contributions to the major emission-lines [e.g., $\text{[O II]}$ $\lambda{3727}Å$, $\text{[O III]}$ $\lambda{5007}Å$, $\text{H}\alpha$] from more than one AGN component. Hence, the chosen SFR indicator must be AGN corrected, tracing the younger, more recent star-forming regions in the host merger. 

A useful optical star formation indicator, which extends to the low-mid redshift range (i.e., ${z}<{0.5}$), is the $\text{[O II]}$ $\lambda{3727}Å$ emission-line. Much like it's more reliable Balmer cousin, the $\text{H}\alpha$ $\lambda 6563Å$ emission-line, the $\text{[O II]}$ $\lambda{3727}Å$ emission-line is widely used as an accurate tracer of star formation in extragalactic surveys [e.g., \cite{Lilly_1996}; \cite{Hippelein_2003}] and it can be an equally effective tracer of star formation in systems containing luminous AGN [\cite{2006ApJ...642..702K}; \cite{2009ApJ...696..396S}; \cite{2012MNRAS.427.2401K}]. Using comparisons of the equivalent width (EW) of the $\text{[O II]}$  $\lambda{3727}Å$ emission-line from a compostie 2dF quasi-stellar object (QSO) spectrum with that of a composite normal galaxy, \cite{2002MNRAS.337..275C} postulated that $\text{[O II]}$ $\lambda{3727}Å$ emission can be attributed to star-formation within the AGN host, valid over a wide range of absolute magnitudes. Hence, although the contribution to $\text{[O II]}$ from the AGN is relatively weak, the nature of the contamination can be extrapolated with a well defined method.   

As mentioned, much like the $\text{H}\alpha$ Balmer emission-line, the $\text{[O II]}$  $\lambda{3727}Å$ line suffers from contamination from the host AGN. The AGN contribution is well-defined for composite AGN-star forming objects: i.e., if high-ionization AGNs play host to substantial levels of ongoing star formation, the integrated contribution from $\text{H II}$ regions will increase the intensity of the $\text{[O II]}$ $\lambda3727Å$  emission-line (compared to the $\text{[O III] }\lambda\lambda4959,50007ÅÅ$ emission-line, where the majority is ascribed to the AGN itself). \cite{2005ApJ...629..680H} discussed the merits of using the $\text{[O II]}$ $\lambda{3727}Å$ emission-line within AGN. Much like the star forming HII regions in galaxies, the narrow-line region of an AGN will also emit this particular line. In narrow-line regions governed by high-ionization parameters, such as those pertinent to Seyfert galaxies, $\text{[O II]}$ $\lambda{3727}Å$ is observed and predicted to be relatively weak: varying from 10--30\% of the $\text{[O III]}$ $\lambda{5007}Å$ emission line [\cite{Ferland_1986}; \cite{Ho_1993}] i.e., $\text{[O II]/[O III]}\approx{0.1-0.3}$. Hence, any deviation excessive from this prediction can be plausibly attributed to star formation within the AGN host galaxy, assuming $\text{[O II]}$ emission in pure AGN is generically quite weak [e.g., \cite{Ferland_1986}; \cite{Ho_1993}; \cite{2006ApJ...642..702K}]. Underlying our work is the assumption that the merger itself triggers both star-formation and two accreting SMBH central to each of the merging pairs; effectively our dual AGN. The method of determining the SFR from the $\text{[O II]}$ $\lambda{3727}Å$ flux corrected for a contribution from both the AGN is equally valid. Complications will arise when analysing the sample holistically given that many of our objects may be at different stages in their merger evolution. An interesting caveat for the analysis outlined by \cite{2006ApJ...642..702K} of the \cite{Zakamska2003} sample is that Type II Seyferts show signs of enhanced star formation. \cite{2006ApJ...642..702K} argued that for the average $L_{\text{[O III]}}$ for the \cite{Zakamska2003} sample (i.e., $L_{\text{[O III]}}\approx{3\times{10^{42}}}$ erg s$^{-1}$) the $\text{[O II]/[O III]}\approx{0.06-0.1}$. However, type 2 AGN have $\text{[O II]/[O III]}$ ratios of $\sim$0.75. Hence, attributing this ratio to purely star-formation within the type 2 AGN host yields an excess [O II] luminosity of $\sim{2\times{10^{42}}}$ erg s$^{-1}$ will yield a star-formation approximately double that of M82 (i.e., $\sim$20M$_{\odot}$ yr$^{-1}$).

As mentioned, further complications to our method allows for corrections for contributions from two AGN. As we are dealing with candidate dAGN objects we find we have both redshifted and blueshifted components to the $\text{[O III] }\lambda\lambda4959,50007ÅÅ$ and Balmer $\text{H}\alpha$ emission-line components. Owing to this, we specify corrected $\text{[O II] }\lambda3727Å$ flux, and further calculations for both AGN contributing to the $\text{[O II]}$: logically, extending the work by \cite{2006ApJ...642..702K} to objects classified as dual AGN. For objects classified as AGN from our BPT analysis, we remove between 10-30\% for both the redshifted and blueshifted \text{[O III]} fluxes from the \text{[O II]} flux, i.e.:
\\
\begin{equation}
\label{eq:KimCorrection}
f_{\text{[O II], SF}}=f_{\text{[O II], SF+AGN}}-\left[(0.2\pm{0.1})(\text{[O III]}_r+\text{[O III]}_b)\right].
\end{equation}
\\

Hence, we can use the existing $\text{[O II]}$ measurements to place limits on the SFRs, employing both the [\cite{Kennicutt_1998}] derivation and, perhaps more importantly, the recent calibration of [\cite{Kewley_2004}, eq. [\ref{eq:Kewley}]], which explicitly takes into account extinction and metallicity corrections and can be applied to near-field (low-$z$) galaxies:
\\
\begin{equation}
\begin{align}
\label{eq:Kewley}
{\text{SFR}}_{\text{[O II]}}\,\,[\text{M}_\odot \text{ yr}^{-1}]=\frac{(9.53\pm{0.51})\times10^{-42}\,\,\text{L}_{\text{[O II]}}\,\,(\text{ergs }\text{s}^{-1})}{(-1.75\pm{0.25})[\log{\text{(O/H)}}+12]+(16.73\pm{2.23})}.
\end{align}
\end{equation}
\\
For our sample redshift range (${{0.01}<{z}<{0.4}}$), we advocated using the \cite{Teplitz_2003} matallicity correction of $\log{\text{(O/H)} +12}\sim{8.90}$ provided by a $\text{[O II]/H}\alpha$ ratio of $\sim{0.83}$. At this stage, the \cite{Kennicutt_1998} estimation for the $\text{SFR}_{\text{[O II]}}$ acts as an initial proxy of the star-formation rate (i.e., it is not corrected for the AGN contribution to the $\text{[O II]}$ flux, metallicity, dust obscuration or reddening and represents a maximum plausible star-formation rate of that object) which can be used to filter those objects within the sample where the maximum $\text{SFR}_{\text{[O II]}}$ would not be significant enough for further scrutiny.

\subsubsection{AGN Contribution Corrections (Composite Galaxies)}

For objects classified as composite galaxies under the Kewley-Kauffmann-Stasi{\'n}ska regime; [see \cite{Kewley_Dopita_Sutherland_Heisler_Trevena_2001}; \cite{Kauffmann_2003}; \cite{Stasinska_2006}], we employed a variation on the corrective method in §3.3 of \cite{Wild_2010}.  