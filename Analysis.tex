\section{Methods, Analysis \& Discussions}
\subsection{Constraining dAGN Star-Formation Rates}
\subsubsection{Population SFR Measurements: $\text{SFR}_{\text{H}\alpha}$ or $\text{SFR}_{\text{[O II]}}$?}

The above mentioned Ge+2012 as/pp-2 AGN and as/pp-A AGN sub-samples forms the basis of our hunt for simultaneous dAGN-starbursts, focusing our attention on understanding the star formation rate (SFR), and by extension the SFRs of host dAGN candidates. Such colliding galaxies often display higher than average $\text{H}\alpha$ $\lambda$6563{\AA } (Balmer) emission [e.g., \cite{Kennicutt_1987}] associated to the recombination radiation formed in the HII regions excited by early-type stars. The primary advantages of this method are its high sensitivity, and the direct coupling between the nebular emission and the massive SFR. The star formation in nearby galaxies can be mapped at high resolution even with small telescopes, and the $\text{H}\alpha$ line can be detected in the redshifted spectra of starburst galaxies to z $\gg$ 2 [e.g., \cite{1997ApJ...477L..29B}]. However, one major problem arises when analysing the $\text{H}\alpha$ emission-line component of composite AGN-starburst objects. Both starbursts and AGN make contributions to (i.e., excite) the $\text{H}\alpha$ emission-line: leading to an inaccurately elevated $\text{SFR}$.

Another useful star formation indicator, which can also extend to the low-mid redshift range, is the $\text{[O II]}$ $\lambda{3727}$ optical emission-line. Much like the $\text{H}\alpha$ emission-line the $\text{[O II]}$ $\lambda{3727}$ emission-line is widely used as an accurate tracer of star formation in extragalactic surveys [e.g., \cite{Lilly_1996}; \cite{Hippelein_2003}] and it can be an equally effective tracer of star formation in systems containing luminous AGN [\cite{2006ApJ...642..702K}]. Although the $\text{[O II]}$ line suffers from contamination by host AGN much like $\text{H}\alpha$ the AGN contribution is well-defined for composite AGN-star forming objects: i.e., if high-ionization AGNs play host to substantial levels of ongoing star formation, the integrated contribution from $\text{H II}$ regions will increase the intensity of the $\text{[O II]}$ $\lambda3727$Å  emission-line (compared to the $\text{[O III]}$ $\lambda5007$Å emission-line, where the majority is ascribed to the AGN itself). Ho 2005 ('\textit{AGNs and Starbursts: What Is the Real Connection?}') recently discussed the merits of using the $\text{[O II]}$ $\lambda{3727}$Å emission-line within AGN. Much like the star forming HII regions in galaxies, the narrow-line region of an AGN will also emit this particular line. In narrow-line regions governed by high-ionization parameters, such as those pertinent to Seyfert galaxies, $\text{[O II]}$ $\lambda{3727}$Å is observed and predicted to be relatively weak: varying from 10--30\% of the $\text{[O III]}$ $\lambda{5007}$Å emission line [\cite{Ferland_1986}; \cite{Ho_1993}] i.e., $\text{[O II]/[O III]}\approx{0.1-0.3}$. Hence, any deviation excessive from this prediction can be plausibly attributed to star formation within the AGN host galaxy, assuming $\text{[O II]}$ emission in pure AGN is generically quite weak. Hence, we can use the existing $\text{[O II]}$ measurements to place limits on the SFRs, employing both the [\cite{Kennicutt_1998}] derivation and, perhaps more importantly, the recent calibration of [\cite{Kewley_2004}, eq. [\ref{eq:Kewley}]], which explicitly takes into account extinction and metallicity corrections:
\\
\begin{equation}
\begin{align}
\label{eq:Kewley}
{\text{SFR}}_{\text{[O II]}}\,\,[\text{M}_\odot \text{ yr}^{-1}]=\frac{(9.53\pm{0.51})\times10^{-42}\,\,\text{L}_{\text{[O II]}}\,\,(\text{ergs }\text{s}^{-1})}{(-1.75\pm{0.25})[\log{\text{(O/H)}}+12]+(16.73\pm{2.23})}.
\end{align}
\end{equation}
\\
