\section{Data Reduction, Analysis \& Discussions}
\subsection{Constraining dAGN Star-Formation Rates}
\subsubsection{Constraining AGN Contributions}

As we have outlined, one of the purposes of this paper is to understand the contribution from our sample of dual-AGN to the $\text{[O II]}$ $\lambda{3727}Å$ (i.e., the $\lambda\lambda{3727}$,$3729ÅÅ$ doublet). \cite{2009ApJ...696..396S} showed that the host galaxies of AGN have ongoing star-formation with a broad range of rates ($\approx{1-100}$M$_{\odot}$ yr$^{-1}$) higher than that of the overall massive ($\log{M}>{10.6}$) galaxy population and essentially equivalent to those forming stars (i.e., emission-line galaxies). 

Even though the production of this low ionization line has been shown to be relatively weak in AGN [e.g., \cite{Ferland_1986}; \cite{Ho_1993}; \cite{2006ApJ...642..702K}].A considered evaluation of the AGN contribution to the [O II] emission is required. We took a sample of 100 local AGN spanning a similar redshift extent as our parent dual AGN sample (i.e., {0.0319}<{z}<{0.6427}).

\subsubsection{AGN Contribution Corrections (Composite Galaxies)}

By their very nature, the star-formation rates of composite emission-line galaxies (Kewley-Kauffmann demarcation region; see gray region: Fig. \ref{BPT-Kewley}) are very difficult to discern from that of their AGN host. For objects classified as composite galaxies under the Kewley-Kauffmann-Stasi{\'n}ska regime; [see \cite{Kewley_Dopita_Sutherland_Heisler_Trevena_2001}; \cite{Kauffmann_2003}; \cite{Stasinska_2006}], we employed a variation on the corrective method in §3.3 of \cite{Wild_2010}. Our method relies on a multi-faceted approach to understanding the AGN contribution to the $\text{[O II]}$ $\lambda3727$ flux from the $\text{[O III]}$ $\lambda5007$ flux, analysing the composite nature of the emission source using BPT diagnostics and then correcting the contribution to the $L_{\text{[O III]}}$ from $\text{[H II]}$ regions in the source. Once composite objects have been ascertained from BPT diagnostics the correction to the $\text{[O II]}$ flux is made by removing the $L_{\text{[O III]}}$ contribution from star-formation regions.
  