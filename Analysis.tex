\section{Methods, Analysis \& Discussions}
\subsection{Constraining dAGN Star-Formation Rates}
\subsubsection{Population SFR Measurements: $\text{SFR}_{\text{H}\alpha}$ or $\text{SFR}_{\text{[O II]}}$?}

Colliding galaxies often display higher than average $\text{H}\alpha$ $\lambda 6563${\AA } (Balmer) emission [e.g., \cite{Kennicutt_1987}] associated to the recombination radiation formed in the HII regions excited by early-type stars. The primary advantages of this method are its high sensitivity, and the direct coupling between the nebular emission and the massive SFR. The star formation in nearby galaxies can be mapped at high resolution even with small telescopes, and the $\text{H}\alpha$ line can be detected in the redshifted spectra of starburst galaxies to z $\gg$ 2 [e.g., \cite{1997ApJ...477L..29B}]. However, one major problem arises when analysing the $\text{H}\alpha$ emission-line component of composite AGN-starburst objects. Both star-forming galaxies and AGN make contributions to (i.e., excite) the $\text{H}\alpha$ emission-line: leading to an inaccurately elevated $\text{SFR}$. 

Determining the star-formation rates (SFRs) of dual AGN host galaxies is a diveresely complex issue. We are potentially measuring the current star-formation rate of a galaxy in mid-collision and with emissivity from more than one central AGN. Any indicator used must be AGN corrected, as well as being a recent tracer of star-forming regions in the host merger. Another useful star formation indicator, which can also extend to the low-mid redshift range, in optical observations is the $\text{[O II]}$ $\lambda{3727}${\AA } emission-line. Much like the $\text{H}\alpha$ $\lambda 6563${\AA } emission-line the $\text{[O II]}$ $\lambda{3727}${\AA } emission-line is widely used as an accurate tracer of star formation in extragalactic surveys [e.g., \cite{Lilly_1996}; \cite{Hippelein_2003}] and it can be an equally effective tracer of star formation in systems containing luminous AGN [\cite{2006ApJ...642..702K}; \cite{2009ApJ...696..396S}; \cite{2012MNRAS.427.2401K}]. Although the $\text{[O II]}$  $\lambda{3727}$Å line suffers from contamination by host AGN much like  $\text{H}\alpha$ $\lambda 6563${\AA } the AGN contribution is well-defined for composite AGN-star forming objects: i.e., if high-ionization AGNs play host to substantial levels of ongoing star formation, the integrated contribution from $\text{H II}$ regions will increase the intensity of the $\text{[O II]}$ $\lambda3727$Å  emission-line (compared to the $\text{[O III]}$ $\lambda5007$Å emission-line, where the majority is ascribed to the AGN itself). \cite{2005ApJ...629..680H} discussed the merits of using the $\text{[O II]}$ $\lambda{3727}$Å emission-line within AGN. Much like the star forming HII regions in galaxies, the narrow-line region of an AGN will also emit this particular line. In narrow-line regions governed by high-ionization parameters, such as those pertinent to Seyfert galaxies, $\text{[O II]}$ $\lambda{3727}$Å is observed and predicted to be relatively weak: varying from 10--30\% of the $\text{[O III]}$ $\lambda{5007}$Å emission line [\cite{Ferland_1986}; \cite{Ho_1993}] i.e., $\text{[O II]/[O III]}\approx{0.1-0.3}$. Hence, any deviation excessive from this prediction can be plausibly attributed to star formation within the AGN host galaxy, assuming $\text{[O II]}$ emission in pure AGN is generically quite weak [e.g., \cite{Ferland_1986}; \cite{Ho_1993}; \cite{2006ApJ...642..702K}]. 

Hence, we can use the existing $\text{[O II]}$ measurements to place limits on the SFRs, employing both the [\cite{Kennicutt_1998}] derivation and, perhaps more importantly, the recent calibration of [\cite{Kewley_2004}, eq. [\ref{eq:Kewley}]], which explicitly takes into account extinction and metallicity corrections and can be applied to near-field (low-$z$) galaxies:
\\
\begin{equation}
\begin{align}
\label{eq:Kewley}
{\text{SFR}}_{\text{[O II]}}\,\,[\text{M}_\odot \text{ yr}^{-1}]=\frac{(9.53\pm{0.51})\times10^{-42}\,\,\text{L}_{\text{[O II]}}\,\,(\text{ergs }\text{s}^{-1})}{(-1.75\pm{0.25})[\log{\text{(O/H)}}+12]+(16.73\pm{2.23})}.
\end{align}
\end{equation}
\\
For our sample redshift range (${{0.01}<{z}<{0.4}}$), we advocated using the \cite{Teplitz_2003} matallicity correction of $\log{\text{(O/H)} +12}\sim{8.90}$ provided by a $\text{[O II]/H}\alpha$ ratio of $\sim{0.83}$. At this stage, the \cite{Kennicutt_1998}, eq. [\ref{eq:Kennicutt1}] acts as an initial estimation of the star-formation rate (i.e., it is not corrected for the AGN contribution to the $\text{[O II]}$ flux, metallicity, dust obscuration or reddening and represents the maximum plausible star-formation rate of that object).

Although using the $\text{H}\alpha$ is a perfectly valid star formation indicator for normal galaxies, it is not valid when discerning the star-formation rate of an active galaxy. However, the $\text{H}\alpha$ emission-line provides a useful upper bound to the SFR (i.e., assuming the AGN plays no contribution to the $\text{H}\alpha$ flux). We can therefore make associative judgements on the validity of the $\text{SFR}_{\text{[O II]}}$ obtained from the $\text{[O II] }\lambda3727$ emission-line: that is to say, any star-formation rate obtained from the $\text{SFR}_{\text{[O II]}}$ that is higher than a theoretical maximum star-formstion rate from the $\text{H}\alpha$ can be judged to be an unreliable, contaminated or skewed result. We should expect that a corrected $\text{SFR}_{\text{[O II]}}$ should be less than a $\text{SFR}$ calculated using the entire flux of the $\text{H}\alpha$. In short, we exclude any values where the upper $\text{SFR}_{\text{[O II]}}$ (corrected for blue- and red-shifted AGN contributions of $0.2\pm{0.1}\times\text{[O III]}$) is less than both the \cite{Kennicutt_1998} calibration and the maximum possible SFR calculated from the $\text{H}\alpha$ flux. We also exclude those where the AGN contribution of $0.3(\text{[O III]})$ results in a $\text{SFR}_{\text{[O II]}}{<}$10 M$_\odot$ yr$^{-1}$.

As mentioned, further complications to our method allows for corrections for contributions from two AGN. As we are dealing with candidate dAGN objects we find we have both redshifted and blueshifted components to the $\text{[O III] }\lambda\lambda4959,50007$ and $\text{H}\alpha$ emission-lines components. Owing to this, we specify corrected $\text{[O II] }\lambda3727$ flux, and further calculations for both AGN contributing to the $\text{[O II]}$: logically, extending the work by \cite{2006ApJ...642..702K} for two dual AGN. We remove between 10-30\% for both the redshifted and blueshifted \text{[O III]} fluxes from the \text{[O II]} flux, i.e.:
\\
\begin{equation}
\label{eq:KimCorrection}
f_{\text{[O II], SF}}=f_{\text{[O II], SF+AGN}}-\left[(0.2\pm{0.1})(\text{[O III]}_r+\text{[O III]}_b)\right].
\end{equation}
\\
An interesting caveat for the analysis outlined by \cite{2006ApJ...642..702K} of the \cite{Zakamska2003} sample is that Type II Seyferts show signs of enhanced star formation. \cite{2006ApJ...642..702K} argued that for the average $L_{\text{[O III]}}$ for the \cite{Zakamska2003} sample (i.e., $L_{\text{[O III]}}\approx{3\times{10^{42}}}$ erg s$^{-1}$) the $\text{[O II]/[O III]}\approx{0.06-0.1}$. However, type 2 AGN have $\text{[O II]/[O III]}$ ratios of $\sim$0.75. Hence, attributing this ratio to purely star-formation within the type 2 AGN host yields an excess [O II] luminosity of $\sim{2\times{10^{42}}}$ erg s$^{-1}$ will yield a star-formation approximately double that of M82 (i.e., $\sim$20M$_{\odot}$ yr$^{-1}$).

\subsubsection{Constraining AGN Contributions}

As we have outlined, one of the purposes of this paper is to understand the contribution from our sample of dual-AGN to the $\text{[O II]}$ $\lambda{3727}$ (i.e., the $\lambda\lambda{3727}$,$3729$ doublet). Hence, a considered evaluation of the AGN contribution to the [O II] emission line is required even though the production of this low ionization line has been shown to be relatively weak in AGN [e.g., \cite{Ferland_1986}; \cite{Ho_1993}; \cite{2006ApJ...642..702K}].
