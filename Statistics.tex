\section{Statistical Analysis}
\subsection{Comparative Analyses}
\subsubsection{Population Results}

In order to better understand the star-formation properties of dual AGN we select three distinct groups to analyse. Firstly, we have our control sample of Type II AGN with $\log{(\text{[O III]}/\text{H}\beta)}>{6.9}$; which we feel can act as archetypal narrow-line AGN sample from which we gain gain context for the Type II candidate dual AGN sample. Coupled to the two main samples, we also add a third: tidal\footnote{\tiny{Objects classified as tidally interacting taken from \cite{Liu_2011}.}} (interacting), binary\footnote{\tiny{kpc-scale binary AGNs of comparable luminosities, with a relative orbital velocity $\gtrsim{150}$ km s$^{-1}$ taken from \cite{Shen_Liu_Greene_Strauss_2011}.}} and visually confirmed dual-cored AGN candidates taken from \cite{Ge_Hu_Wang_Bai_Zhang_2012}, and that coincide and can be cross-referenced with objects within our Liu+2010 parent dual AGN sample. With this in mind, we firstly look towards the amount of the $\text{[O II]}$ $\lambda\lambda$$3727.09$, $3729.88$\AA\AA{} and $\text{H}\alpha$ integrated flux (in units of $1\text{E}^{-17}$ erg s$^{-1}$ cm$^{-2}$) relative to the integrated flux of the $\text{[O III]}$ $\lambda$$5008.24$\AA. These ratios are important in both our understanding of the star-formation properties of the host galaxies of dual AGN, and the contamination of these particular-emission lines from the dual AGN contribution.

Our analysis of the $\text{[O II]/[O III]}$ ratio for single Type II AGN sample yielded the following statistical results: a 2nd quartile i.e., median, $\tilde{x}$ of 0.2318; standard deviation $\sigma_{X}$ of 0.1402 with a cleaned mean, $\mu$, (i.e., excluded outliers within one standard deviation of the median) of 0.1727 and an uncleaned mean of 0.2681. We also report an average absolute deviation around the median of 0.1030, and minimum and maximum values of 0.0836 and 0.7053, respectively. For a desired confidence level of 95\% for the mean quoted, the corresponding confidence interval is ${0.2403}<{\mu}<{0.2956}$. However, if we exclude all \textit{Tukey outliers} (i.e., those objects which are greater than 1.5 times the interquartile range (IQR) above the third quartile or below the first quartile) for this sample we can quote a 95\% confidence level of ${0.2311}<{\mu}<{0.2809}$.

We applied the same statistical analyses for our Liu+2010 candidate dual AGN sample and to our tidal, binary and dual-cored AGN candidates. Firstly, our analysis of the Type II AGN with double-peaked emission-lines sample yielded the following statistical results: a $\tilde{x}$ of 0.3354; $\sigma_{X}$ of 0.2541 with a $\mu_{c}$ of 0.2771 and a $\mu$ of 0.3988. We also report an average absolute deviation around the median of 0.1450, and minimum and maximum values of 0.1125 and 2.1492, respectively. For a desired confidence level of 95\% for the mean quoted, the corresponding confidence interval is ${0.3584}<{\mu}<{0.4407}$. Further to our statistical analyses for our Type II AGN with double-peaked emission-lines sample, if we were to isolate those objects within our parent sample designated as tidally interacting, binary or dual-cored AGN we quote a $\tilde{x}$ of 0.4594; $\sigma_{X}$ of 0.3041 with a $\mu_{c}$ of 0.3504 and a $\mu$ of 0.5241. We also report an average absolute deviation around the median of 0.2110, with minimum and maximum values of 0.2319 and 1.3832, respectively. Again, for a desired confidence level of 95\% for the mean quoted, the corresponding confidence interval is ${0.3775}<{\mu}<{0.6707}$.

Comaparing all three samples it is evident that within the candidate dual AGN sample there are objects which have elevated $\text{[O II]}$ emission relative to our pure Type II AGN control sample, where our dual AGN which are classified in the literature as either tidally interacting, binary or dual-cored are again showing signs of elevated $\text{[O II]}$ emission relative to both the Type II AGN and the candidate Type II dual AGN samples.
 
\subsubsection{Kolmogorov-Smirnov (K-S) Test between Type II AGN and the candidate Type II dual AGN.}
Given the uncertainties and the spread in the value for the lines luminosity, we employ a statistical tool to confirm that the candidate Type II dual AGN and the normal Type II AGN belong to two different populations, each own with its characteristics. A Kolmogorov-Smirnov (K-S) Test is the best statistical method to evaluate this. It tests if two data sample come from the same underlying distribution. We perform the K-S test by using an ad hoc python program.\\
In Figure 4 we can see the K-S test performed for the [OII]/[OIII] line ratio. To reject the null hypothesis to a 0.001 level, the K-S statistic, D must be
\begin{equation}
D_{n_1,n_2}>1.98\sqrt{{n_1+n_2}\over {n_1n_2}}
\end{equation}
\begin{equation}
D_{n_1,n_2}> 
\end{equation}

has a value of $0.38$ 


with a p-value of $5.49\times10^{-8}$. This is a clear indication that the two samples came from two different populations. This result is important because it clearly 
  
  
  
  
  
  
  