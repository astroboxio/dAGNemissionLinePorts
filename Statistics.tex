\section{Statistical Analysis}
\subsection{Comparative Analyses}
\subsubsection{Population Results}

In order to better understand the star-formation properties of dual AGN we select three distinct groups to analyse. Firstly, we have our control sample of Type II AGN with $\log{(\text{[O III]}/\text{H}\beta)}>{6.9}$; which we feel can act as archetypal narrow-line AGN sample from which we gain gain context for the Type II candidate dual AGN sample. Coupled to the two main samples, we also add a third: tidal (interacting), binary and dual-cored AGN candidates that we take from the literature, and that coincide and can be cross-referenced with objects within the Liu+2010 parent dual AGN sample. With this in mind, we firstly look towards the amount of the $\text{[O II]}$ $\lambda\lambda$$3727.09$, $3729.88$\AA\AA{} and $\text{H}\alpha$ integrated flux (in units of erg s${^-1}$ cm$^{-2}$) relative to the integrated flux of the $\text{[O III]}$ $\lambda$$5008.24$\AA. These ratios are important in both our understanding of the star-formation properties of the host galaxies of dual AGN, and the contamination of these particular-emission lines from the dual AGN contribution.

Our analysis of the $\text{[O II]/[O III]}$ ratio for single Type II AGN sample yielded the following statistical results: 2nd quartile i.e., median, $\tilde{x}$ of 0.2318; standard deviation $\sigma_{X}$ of 0.1402 with a cleaned mean, $\mu$, (i.e., excluded outliers within one standard deviation of the median) of 0.1727 and an uncleaned mean of 0.2681. We also report an average absolute deviation around the median of 0.1030, and minimum and maximum values of 0.0836 and 0.7053, respectively. For a desired confidence level of 95\% for the mean quoted, the corresponding confidence interval is ${0.2403}<{\mu}<{0.2956}$. However, if we again exclude the outliers (i.e., those objects which are greater than 1.5 times the interquartile range (IQR) above the third quartile or below the first quartile) for this sample we can quote a 95\% confidence level of ${0.2311}<{\mu}<{0.2809}$. 

\subsubsection{Kolmogorov-Smirnov Test}
  
  