\section{Methods \& Constraints}
\subsection{SFR Indicators}

\subsubsection{Star-Formation Within AGN}

Perhaps the biggest hurdle facing the evolutionary properties of AGN is quantifying their current star-formation rate. There are numerous star-formation rate indicators that can be applied to normal galaxies [e.g., optical tracers: \cite{Kennicutt_1998}; X-ray tracers: \cite{Gilfanov_2004}]. However, determining the star-formation rates (SFRs) of dual AGN host galaxies is a diversely complex issue. Our primary obstacle is that when analysing the SFR of an AGN one must account for contributions to the flux of our chosen SFR tracer from both the star-forming regions and the AGN itself. For example, both the hard X-ray luminosity, $L_{\text{X}}$ i.e., > 10 keV, indicator proposed by \cite{Gilfanov_2004}, nebular and Balmer recombination lines such as $\text{H}\alpha$ and $\text{H}\beta$ and mid-UV continuum strength measurements (${2000}Å<{\lambda}<3000Å$) would lead to unreliable and elevated results since such emissions are ubiquitous within AGN. Equally futile for AGN, radio synchrotron emission at 21cm is often used as a star formation tracer [\cite{1983A&A...120..219K}], but it is rendered useless, even radio-quiet AGN, because AGNs are never totally radio-silent. Seemingly, any star-formation tracer used in AGN need to be understood in terms of the contamination from the AGN. If this could be quantified in some way, then we could garnish a more reliable measurement of the current star-formation within AGN. Added to this, the majority of objects within the chosen dual AGN sample are potentially at peak luminosity; or somewhere later in their evolution (e.g., ${\tau}>{1.0}$ Gyr), perhaps nearer to the coalescence stage of their SMBH binary pairs ($\lesssim{10}$ kpc). As such, we expect contributions to the major emission-lines [e.g., $\text{[O II]}$ $\lambda{3727}Å$, $\text{[O III]}$ $\lambda{5007}Å$, $\text{H}\alpha$] from more than one AGN component. Hence, the chosen SFR indicator must be AGN corrected, tracing the younger, more recent star-forming regions in the host merger.

\subsubsection{$\text{H}\alpha$ vs. $\text{[O II]}$}

As mentioned, we have two optical possibilities for measuring the star-formation rate of our candidate dual AGN sample, that is the nebular recombination lines of $\text{H}\alpha$ and $\text{[O II]}$. Spectroscopic surveys of distant galaxies [e.g., \cite{Lilly_1996}; \cite{Hippelein_2003}; \cite{Cardiel_2003}], particularly in the intermediate ${0.4}<{z}<{1.0}$ redshift range, routinely use $\text{[O II]}$ $\lambda{3727}Å$ as a tracer of the current SFR due to the $\text{H}\alpha$ $\lambda 6563Å$ Balmer emission-line being outside of the range of optical surveys at such large distances. The $\text{[O II]}$ $\lambda{3727}Å$ emission-line is prominent emission line emanating from low-density, partially ionised $\text{H II}$ nebula regions. As this emission line has been shown to be relatively weak in AGN [e.g., \cite{Ferland_1986}; \cite{Ho_1993}; \cite{Kim_2006}] it can become quite an effective tracer of star-formation in our candidate dual AGN sample. Hence, the $\text{[O II]}$ emission can be an equally effective tracer of star formation in systems containing luminous AGN [\cite{2006ApJ...642..702K}; \cite{2009ApJ...696..396S}; \cite{2012MNRAS.427.2401K}]. Using comparisons of the equivalent width (EW) of the $\text{[O II]}$  $\lambda{3727}Å$ emission-line from a compostie 2dF quasi-stellar object (QSO) spectrum with that of a composite normal galaxy, \cite{2002MNRAS.337..275C} postulated that $\text{[O II]}$ $\lambda{3727}Å$ emission can be attributed to star-formation within the AGN host, valid over a wide range of absolute magnitudes. Hence, although the contribution to $\text{[O II]}$ from the AGN is relatively weak, the nature of the contamination can be extrapolated with a well defined method. However, the $\text{[O II]}$ $\lambda{3727}Å$ emission-line suffers from sensitivity to dust extinction and metallicity effects. For example, comparisons of SFRs derived from [O II] versus SFRs derived from radio and the far-IR continuum, that while dust extinction is certainly non-negligible, on average $\text{[O II]}$ $\lambda{3727}Å$ only suffers from an extinction of $A_V\approx{1}$ mag. Couple to this, \cite{Kewley_2004} have evaluated the influence of metallicity variations on the SFR. They have concluded that this effect is less serious than dust extinction; with plausible metallicity uncertainties of a factor of 2, for example, result in SFR variations of only $\sim$50\%.

\subsection{Constraining AGN Contributions}
\subsubsection{Theoretical AGN Constraints}

As mentioned, much like the $\text{H}\alpha$ Balmer emission-line, the $\text{[O II]}$ $\lambda{3727}Å$ line suffers from contamination from the host AGN. The AGN contribution is well-defined for composite AGN-star forming objects: i.e., if high-ionization AGNs play host to substantial levels of ongoing star formation, the integrated contribution from $\text{H II}$ regions will increase the intensity of the $\text{[O II]}$ $\lambda3727Å$  emission-line (compared to the $\text{[O III] }\lambda\lambda4959,50007ÅÅ$ emission-line, where the majority is ascribed to the AGN itself). \cite{2005ApJ...629..680H} discussed the merits of using the $\text{[O II]}$ $\lambda{3727}Å$ emission-line within AGN. Much like the star forming HII regions in galaxies, the narrow-line region of an AGN will also emit this particular line. In narrow-line regions governed by high-ionization parameters, such as those pertinent to Seyfert galaxies, $\text{[O II]}$ $\lambda{3727}Å$ is observed and predicted to be relatively weak: varying from 10--30\% of the $\text{[O III]}$ $\lambda{5007}Å$ emission line [\cite{Ferland_1986}; \cite{Ho_1993}] i.e., $\text{[O II]/[O III]}\approx{0.1-0.3}$. Hence, any deviation excessive from this prediction can be plausibly attributed to star formation within the AGN host galaxy, assuming $\text{[O II]}$ emission in pure AGN is generically quite weak [e.g., \cite{Ferland_1986}; \cite{Ho_1993}; \cite{2006ApJ...642..702K}]. Underlying our work is the assumption that the merger itself triggers both star-formation and two accreting SMBH central to each of the merging pairs; effectively our dual AGN. The method of determining the SFR from the $\text{[O II]}$ $\lambda{3727}Å$ flux corrected for a contribution from both the AGN is equally valid. Complications will arise when analysing the sample holistically given that many of our objects may be at different stages in their merger evolution. An interesting caveat for the analysis outlined by \cite{2006ApJ...642..702K} of the \cite{Zakamska2003} sample is that Type II Seyferts show signs of enhanced star formation. \cite{2006ApJ...642..702K} argued that for the average $L_{\text{[O III]}}$ for the \cite{Zakamska2003} sample (i.e., $L_{\text{[O III]}}\approx{3\times{10^{42}}}$ erg s$^{-1}$) the $\text{[O II]/[O III]}\approx{0.06-0.1}$. However, type 2 AGN have $\text{[O II]/[O III]}$ ratios of $\sim$0.75. Hence, attributing this ratio to purely star-formation within the type 2 AGN host yields an excess [O II] luminosity of $\sim{2\times{10^{42}}}$ erg s$^{-1}$ will yield a star-formation approximately double that of M82 (i.e., $\sim$20M$_{\odot}$ yr$^{-1}$).

As mentioned, further complications to our method allow for corrections for contributions from two AGN. As we are dealing with candidate dAGN objects we find we have both redshifted and blueshifted components to the $\text{[O III] }\lambda\lambda4959,50007ÅÅ$ and Balmer $\text{H}\alpha$ emission-line components. Owing to this, we specify corrected $\text{[O II] }\lambda3727Å$ flux, and further calculations for both AGN contributing to the $\text{[O II]}$: logically, extending the work by \cite{2006ApJ...642..702K} to objects classified as dual AGN. For objects classified as AGN from our BPT analysis, we remove between 10-30\% for both the redshifted and blueshifted \text{[O III]} fluxes from the \text{[O II]} flux, i.e.:
\\
\begin{equation}
\label{eq:KimCorrection}
f_{\text{[O II], SF}}=f_{\text{[O II], SF+AGN}}-\left[(0.2\pm{0.1})(\text{[O III]}_r+\text{[O III]}_b)\right].
\end{equation}
\\
Although the SDSS DR7 and DR10 emission-line analyses are valid for most objects, and for the AGN sub-class, we find that not all AGN designations are correct or, indeed, accurate. Instead, an empirically derived $\text{[O II]}/\text{[O III]}$ extent for pure narrow-line Type II AGN was measured from our control sample of 100 narrow-line AGN selected from the SDSS with the criteria outlined in §2.1.2 of this paper. From that we can remove the AGN contribution to the integrated $\text{[O II]}$ flux according to this empirically derived ratio.

Using our $\text{[O II]}$ measurements to place limits on the SFRs, employing both the [\cite{Kennicutt_1998}] derivation and, perhaps more importantly, the recent calibration of [\cite{Kewley_2004}, eq. [\ref{eq:Kewley}]], which explicitly takes into account extinction and metallicity corrections and can be applied to near-field (low-$z$) galaxies:
\\
\begin{equation}
\begin{align}
\label{eq:Kewley}
{\text{SFR}}_{\text{[O II]}}\,\,[\text{M}_\odot \text{ yr}^{-1}]=\frac{(9.53\pm{0.51})\times10^{-42}\,\,\text{L}_{\text{[O II]}}\,\,(\text{ergs }\text{s}^{-1})}{(-1.75\pm{0.25})[\log{\text{(O/H)}}+12]+(16.73\pm{2.23})}.
\end{align}
\end{equation}
\\
For our sample redshift range (${{0.01}<{z}<{0.4}}$), we advocated using the \cite{Teplitz_2003} matallicity correction of $\log{\text{(O/H)} +12}\sim{8.90}$ provided by a $\text{[O II]/H}\alpha$ ratio of $\sim{0.83}$. At this stage, the \cite{Kennicutt_1998} estimation for the $\text{SFR}_{\text{[O II]}}$ acts as an initial proxy of the star-formation rate (i.e., it is not corrected for the AGN contribution to the $\text{[O II]}$ flux, metallicity, dust obscuration or reddening and represents a maximum plausible star-formation rate of that object) which can be used to filter those objects within the sample where the maximum $\text{SFR}_{\text{[O II]}}$ would not be significant enough for further scrutiny.

\subsubsection{Empirical AGN Constraints (Pure Type II AGN)}

Utilising Baldwin, Phillips \& Terlevich (BPT) diagnostics of AGN and star-forming galaxies, it may be possible to empirically constrain the $\text{[O II]}/\text{[O III]}$ ratio of `pure' Seyfert II AGN by defining the nature of `pure' narrow-line AGN. Since the majority of our sample Firstly, we place a lower limit on the BPT $\text{[O III]}/\text{H}\beta$ ratio of $\log(\text{[O III]}/\text{H}\beta)>{0.690}$ for our control sample. Thus, even if there were star-formation present within the purest AGN we are still able to place a constraint on this ratio relative to a pure sample. In essence, understanding the extent of the $\text{[O II]}$ emission relative to the $\text{[O III]}$ for pure AGN for a statistically significant control sample it might be possible to understand and constrain how much star-formation is present in actively accreting dual AGN or merging AGN pairs. We can further demarcate our samples according to both $\text{H}\alpha/\text{H}\beta$ Balmer decrement, as well as the luminosity of the $\text{[O III]}$ $\lambda{5007}Å$ emission-line, i.e., $L_{\text{[O III]}}$, and the SDSS r band magnitude. From this we can calculate the excess, normality or scarcity of $\text{[O II]}$ emission in candidate dual AGN relative to such a control sample based on the comparative properties listed.

\subsubsection{Candidate dual AGN Samples: Limitations}
\subsubsection{Modified AGN Optical Line Diagnostics \& $D_{n}(4000)$ vs. $\log(SFR/M_{*})$}

In order to test our method and to further constrain the nature of our candidate dual AGN sample we employ the use of our own modified DOII diagram (for the DEW diagnsotic see \cite{Stasinska_2006}) which plots the measured $D_{n}(4000)Å$ break, i.e., the ratio of the flux in the red continuum (4001--4103 Å) to that in the blue continuum (3849--3952 Å) [see \cite{Balogh_1999}], against $log(\text{[O II]}/\text{[O III]})$. \cite{Marocco_2011} used the DEW diagnostic to unambiguously demarcate normal star-forming galaxies from Seyfert 2. In a similar fashion we aim to understand the differences between the observed $D_{n}(4000)Å$ break and the strength of the $\text{[O II]}$ $\lambda{3727}Å$ ratio for both our Type II AGN control sample and our dual AGN candidate sample. 
  
  
  